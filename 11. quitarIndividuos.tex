
% === quitarIndividuos === %
\subsection{proc. quitarIndividuos}
    
    \subsubsection{Especificaci\'on:}    
        
        \begin{proc}{quitarIndividuos}{\Inout th : $eph_{h}$,\ \Inout ti : $eph_{i}$,\ \In busqueda : \TLista{(ItemIndividuo,\ dato)},\ \Out result : ($eph_{h}$,\ $eph_{i}$)}{}
            \pre{
                sonTablasValidas(th,\ ti)\ \y\ esBusquedaValida(busqueda)\ \y\ th = th_{0}\ \y\ ti = ti_{0}
            }
            \post{\\
                (tienenLosMismosElementos(th_{0},\ th ++\ result_{0})\ \y\\
                \tab tienenLosMismosElementos(ti_{0},\ ti ++\ result_{1}))\ \yLuego\\
                (losIndividuosEstanFiltrados(ti_{0},\ ti,\ result_{1},\ busqueda)\ \y\\ 
                \tab losHogaresEstanFiltrados(th_{0},\ th,\ result_{0},\ ti_{0},\ busqueda))\\
            }
        \end{proc}

    \subsubsection{Predicados y funciones auxiliares:}
        
        \noindent\pred{esBusquedaValida}{Q : \TLista{(ItemIndividuo,\ dato)}}{ 
            \comment{\tab}{$Q$ = query}
            % del mismo modo que no hace falta validar que un argumento se un entero, no hace falta validar que es un itemIndividuo
            \tab(\forall i : \ent)(0 \leq i < |Q| \implicaLuego (\\
            \tab\tab pideUnDatoValido(Q[i])\ \y\
            \neg(\exists j : \ent)((0 \leq j < |Q|\ \y\ i \neq j)\ \yLuego\ (Q[i])_{0} = (Q[j])_{0}))\\
            \tab)\\
        }
        $\newline$
        \pred{pideUnDatoValido}{filtro : $(ItemIndividuo,\ dato)$}{\\
            \tab (filtro_{0} = indcodusu\ \y\ 0 \leq filtro_{1})\ \vee\\
            \tab (filtro_{0} = componente\ \y\ 1 \leq filtro_{1} \leq 20)\ \vee\\
            \tab (filtro_{0} = inda\tilde{n}o\ \y\ 1810 \leq filtro_{1})\ \vee\\
            \tab (filtro_{0} = indtrimestre\ \y\ 1 \leq filtro_{1} \leq 4)\ \vee\\
            \tab (filtro_{0} = ch4\ \y\ 1 \leq filtro_{1} \leq 2)\ \vee\\
            \tab (filtro_{0} = ch6\ \y\ 0 \leq filtro_{1})\ \vee\\
            \tab (filtro_{0} = nive\_ed\ \y\ 0 \leq filtro_{1} \leq 1)\ \vee\\
            \tab (filtro_{0} = estado\ \y\ -1 \leq filtro_{1} \leq 1)\ \vee\\
            \tab (filtro_{0} = cat\_ocup\ \y\ 0 \leq filtro_{1} \leq 4)\ \vee\\
            \tab (filtro_{0} = p47t\ \y\ -1 \leq filtro_{1})\ \vee\\
            \tab (filtro_{0} = ppo4g\ \y\ 1 \leq filtro_{1} \leq 10)\\
        }
        %$\newline$
        %\pred{esParticion}{original,\ s$_{1}$,\ s$_{2}$ : \TLista{T}}{\\
            %\tab |original| \geq |s_{1}|\ \y\ |s_{2}| = |original| - |s_{1}|\ \y\\
        %    \tab(\forall i:T)(i \in s_{1} ++\ s_{2}\ \iff i \in original)\\ 
            %\y\ \neg(\exists i : T)(i \in s_{1}\ \y\ i \in s_{2})\\ 
        %}
        $\newline$
        \pred{losIndividuosEstanFiltrados}{original,\ filtrada,\ complemento : $eph_{i}$,\ Q : \TLista{(ItemIndividuo,\ dato)}}{\\
        \tab(\forall i : $individuo$)(i \in original\ \implicaLuego\ (\\
        \tab\tab (i \in complemento\ \y\ i \notin filtrada)\ \iff esBusquedaExitosa(i,\ Q)\\
        \tab))\\
        }
        $\newline$
        \pred{losHogaresEstanFiltrados}{original,\ filtrada,\ complemento : $eph_{h}$,\ ti : $eph_{i}$,\  Q : \TLista{(ItemIndividuo,\ dato)}}{\\
        \tab(\forall h : $hogar$)(h \in original\ \implicaLuego\ (\\
        \tab\tab (h \in complemento\ \y\ h \notin filtrada)\ \iff\\
        \tab\tab (\forall i : $individuo$)((i \in ti\ \y\ i[@indcodusu] = h[@hogcodusu])\ \implicaLuego\ (\\
        \tab\tab\tab  esBusquedaExitosa(i,\ Q)\\
        \tab\tab))\\
        \tab))\\
        }
        $\newline$
        \pred{esBusquedaExitosa}{i : $individuo$,\ Q : \TLista{(ItemIndividuo,\ dato)}}{\\
            \tab (\forall\ filtro : (ItemIndividuo,\ dato))(filtro \in Q\ \implicaLuego\\
            \tab\tab i[itemIndividuo.ord(filtro_{0})] = filtro_{1}\\ 
            \tab)\\
        }    
    
    
    \subsubsection{Observaciones:}
        \begin{itemize}
            \item se hace uso del predicado $tienenLosMismosElementos$ definido en 2.1.
            \item el predicado $losHogaresEstanFiltrados$ considera que un hogar debe ser filtrado sólo si todos los individuos
            que viven en él son filtrados.
            \item decidimos incorporar directamente el predicado $pideUnDatoValido$ en esta sección, en vez de modularizar los predicados
            $attEnRango$ definidos en 1.1.2., porque -para el alcance de este TPE- consideramos que complejizaría 
            la lectura de ambos procedimientos en mayor medida de lo que les podría aportar.
        \end{itemize}