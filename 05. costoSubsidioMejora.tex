% === costoSubsidioMejora === % 
\subsection{proc. costoSubsidioMejora}

    \subsubsection{Especificaci\'on:}    
        \begin{proc}{costoSubsidioMejora}{\In th: $eph_{h}$, \In ti: $eph_{i}$, \In monto: \ent, \Out res: \ent}{}
            \pre{sonEncuestasValidas(th,\ ti)\ \y\ monto > 0} 
            \post{res = monto * \displaystyle\sum_{h \in th}(\IfThenElse{esHogarValido_{1{.}5}(ti,\ h)}{1}{0})}    
        \end{proc}

    \subsubsection{Predicados y funciones auxiliares:}
        \noindent\pred{esHogarValido$_{1{.}5}$}{ti: $eph_{i}$, h: $hogar$}{\\
            \tab h[@ii7] = 1\ \y\ h[@iv1] = 1\ \y\ \#individuosEnHogar(ti,\ h[@hogcodusu]) - 2 > h[@ii2]\\
        }

    \subsubsection{Observaciones:}
        \begin{itemize}
            \item consideramos que un subsidio es necesariamente un monto positivo y que, dado el objetivo final de la especificaci\'on
            debe ser mayor a 0.
        \end{itemize}
