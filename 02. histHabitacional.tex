% === histHabitacional === % 
\subsection{proc. histHabitacional}

    \subsubsection{Especificaci\'on:}
        \begin{proc}{histHabitacional}{\In th: $eph_{h}$, \In ti: $eph_{i}$, \In region: $dato$, \Out res: \TLista{\ent}}{}
        \pre{sonTablasValidas(th,\ ti)\ \y\ 1 \leq region \leq 6\ \y\ (\exists h: hogar)(laCasaEstaEnLaRegion(th,\ h,\ region))}
        % CORREGIR
        \post{\\
            maximoDeHabitaciones(th,\ region, res)\ \y \\
            (\forall i:\ent)(0\leq i < |res|\ \implicaLuego\\ 
            \tab res[i] = \#casasPorNroDeHabitaciones(th,\ k,\ i + 1)\\
        )}
        \end{proc}

    \subsubsection{Predicados y funciones auxiliares:}
        \noindent\pred{laCasaEstaEnLaRegion}{th: $eph_{h}$, h: $hogar$, region: $dato$}{\\
            \tab h \in th\ \yLuego\ esHogarValidoParaHistograma(h,\ region)\\
        }
        $\newline$
        \noindent\pred{esHogarValidoParaHistograma}{h: $hogar$, region: $dato$}{\\
            \tab h[@region] = region\ \y\ h[@iv1] = 1\\
        }
        $\newline$
        \noindent \pred{maximoDeHabitaciones}{th: $eph_{h}$, region: $dato$, res: \TLista{\ent}}{\\
            \tab(\exists h : hogar)(laCasaEstaEnLaRegion(th,\ h,\ region)\ \yLuego\ (\\
            \tab\tab h[@iv2] = |res|\ \y\ (\forall h_2 : hogar)(laCasaEstaEnLaRegion(th,\ h_2,\ region)\ \implicaLuego\ h[@iv2] \geq h_2[@iv2])\\
            \tab)\\
        }
        $\newline$
        \noindent\aux{$\#$casasPorNroDeHabitaciones}{th: $eph_{h}$, region: $dato$, habitaciones: \ent}{\ent}{\\[2ex]
            \tab\displaystyle\sum_{h \in th}
            {(\IfThenElse {esHogarValidoParaHistograma(h,\ region)\ \y\ h[@iv2] = habitaciones}{1}{0})}
        }

    \subsubsection{Observaciones:}
        \begin{itemize}
            \item se hace uso del predicado $sonTablasValidas$ definido en 1.1.2.
            \item consideramos, mediante el predicado $laCasaEstaEnLaRegion$ en la precondici\'on, que no tiene sentido preguntarse sobre el 
            histograma habitacional de una regi\'on si \'esta no tiene hogares.
            \item el predicado $maximoDeHabitaciones$ verifica que el largo de la resolución corresponda con la cantidad 
            máxima de habitaciones en la tabla de hogares.
        \end{itemize}

