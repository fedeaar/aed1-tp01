%Ejercicio 4

\begin{proc}{creceElTeleworkingEnCiudadesGrandes}{\In t1h: $eph_{h}$, \In t1i: $eph_{i}$, \In t2h: $eph_{h}$, \In t2i: $eph_{i}$, \Out res: Bool}{}
\pre{\\
    validarEncuesta(t1h,tli) \y validarEncuesta(t2h, t2i) \yLuego\\
    %Filtro de mismo trimestre y diferentes años
    %Esto seria si se acepta mas de un año de diferencia
    t1h[0][@hoga\tilde{n}o] < t2h[0][@hoga\tilde{n}o] \y \\
    t1h[0][@hogtrimestre] = t2h[0][@hogtrimestre]\\
    %Si quisieramos hacer con solo un año
    %t1h[0][@hogaño] = t2h[0][@hogaño] - 1 ^  t1h[0][@hogtrimestre] = t2h[0][@hogtrimestre]
}
\post{
    res = percentTeleworking(t1h,t1i) < percentTeleworking(t2h,t2i)
}
\end{proc}

\pred{filtroTeleworking}{thr: hogar}{
    %Si es algomerado de >500.000 habitantes
    thr[@mas\_500] = 1 \y\\ 
    %Si es casa/departamento
    (thr[@iv1] = 1 \vee thr[@iv1] = 2)
}

\pred{hacenTeleworking}{th: $eph_{h}$, p: individuo}{
    %Filtro de individuos que hacen teleworking
    filtroTeleworking(th[indiceCasaDePersona(th,p[@indcodusu])]) \y th[j][@ii3] = 1 \y p[@ppo4g] = 6
}


\pred{cumpleFiltroTeleworking}{th: $eph_{h}$, p: individuo}{
    %Filtro de la gente en general
    filtroTeleworking(th[indiceCasaDePersona(th,p[@indcodusu])])
}

\aux{indiceCasaDePersona}{th: $eph_{h}$, indcodusu: \ent}{\ent}{\\
    %Busca indice de hogar de una persona en base a su codigo indcodusu
    \sum_{j=0}^{|th|-1}\IfThenElse{th[j][@hogcodusu] = indcodusu}{j}{0}
}

\aux{percentTeleworking}{th: $eph_{h}$, ti: $eph_{i}$}{\float}{\\
    %Si th/ti son válidos
    %Agregue varios predicados para que no me quede choclazo, pero la verdad son medio irredundantes. Fijate que te parecen y decime!
    (\sum_{j=0}^{|ti|-1}(\IfThenElse{hacenTeleworking(th,ti[j])}{1}{0})) /\\
    (\sum_{j=0}^{|ti|-1}(\IfThenElse{cumpleFiltroTeleworking(th,ti[j]}{1}{0})
}

%Ejercicio5
\begin{proc}{costoSubsidioMejora}{\In th: $eph_{i}$, \In ti: $eph_{i}$, \In monto: \ent, \Out res: \ent}{}
\pre{
    %Encuestas y montos deben ser válidos    
    validarEncuesta(th,ti) \y monto \geq 0
} 
\post{ 
    res = monto * \sum_{j=0}^{|th|-1}(\IfThenElse{filtroSubsidio(ti,th[j])}{1}{0})
}
\end{proc}
 
\pred{filtroSubsidio}{ti: $eph_{i}$, thr: hogar}{
    %Para no olvidarme, agregue que hogar tiene que ser una casa y la lista de individuos a la funcion auxiliar
    (thr[@ii7] = 1) \y (thr[@iv1] = 1) \y\\
    individuosEnHogar(ti,thr[@hogcodusu]) - 2 > thr[@ii2]
}
