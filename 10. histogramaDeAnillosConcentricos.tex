% === histogramaDeAnillosConcentricos === %
\subsection{proc. histogramaDeAnillosConcentricos}
    \subsubsection{Especificaci\'on:}    
        \begin{proc}{histogramaDeAnillosConcentricos}{\In th: $eph_{h}$, \In centro: \ent$\times$\ent, \In distancias: \TLista{\ent}, \Out res: \TLista{\ent}}{}
            \pre{esTablaDeHogaresValida(th)\ \y\ esCentroValido(centro)\ \y\ sonDistanciasValidas(distancias)} 
            \post{\\
            |result| = |distancias|\ \yLuego (\\
            result[0] = \#HogaresEnAnillo(th,\ centro,\ 0,\ distancias[0])\ \y\\
            (\forall i:\ent)(0 < i < |result|\ \implicaLuego\\ 
                \tab result[i] = \#HogaresEnAnillo(th,\ centro,\ distancias[i - 1],\ distancias[i])\\
            ))} 
        \end{proc}
    
    \subsubsection{Predicados y funciones auxiliares:}
        \noindent\pred{esCentroValido}{centro \ent$\times$\ent}{\\
        \tab -90 \leq centro_{0} \leq 90\ \y\ -180 \leq centro_{1} \leq 180\\
        }
        $\newline$
        \noindent\pred{sonDistanciasValidas}{distancias: \TLista{\ent}}{\\
            \tab|distancias| > 0\ \yLuego\ (distancias[0] > 0\ \y\ (\forall i:\ent)(0 \leq i < |distancias| - 1\ \implicaLuego\ distancias[i] < distancias[i + 1]))\\
        }
        $\newline$
        \aux{$\#$HogaresEnAnillo}{th :  $eph_{h}$, centro: \ent$\times$\ent, desde: \ent, hasta: \ent}{\ent}{\\[2ex]
            \tab\displaystyle\sum_{h \in th}\IfThenElse{cuadrado(desde) \leq distancia(h,\ centro) < cuadrado(hasta)}{1}{0}
        }
        $\newline$
        \aux{distancia}{h: $hogar$, centro: \ent$\times$\ent}{\float}{
            cuadrado(h[@hoglatitud] - centro_{0}) + cuadrado(h[@hoglongitud] - centro_{1})
        }
        $\newline$
        \aux{cuadrado}{n : \ent}{\ent}{n * n}

    \subsubsection{Observaciones:}
        \begin{itemize}
            \item se hace uso del predicado $esTablaDeHogaresValida$ definido en 1.1.2.
            \item Dado que la pertenencia de una distancia P = (x, y) a un anillo concéntrico definido en el intervalo (positivo) de radios [A, B) respecto al centro C = (x$_{0}$, y$_{0}$) se define como: 
                \begin{equation}
                    A\ \leq\ \sqrt{(x - x_{0})^{2} + (y - y_{0})^{2}}\ <\ B
                \end{equation}
                Por simple manipulación algebráica (elevando al cuadrado), la misma relación se mantiene para:
                \begin{equation}
                    A^{2}\ \leq\ (x - x_{0})^{2} + (y - y_{0})^{2}\ <\ B^{2}
                \end{equation}
                el predicado $\#HogaresEnAnillo$ hace uso de esta observaci\'on.
                
                % agregar que todo es nec positivo
                % y otras obs
        \end{itemize}