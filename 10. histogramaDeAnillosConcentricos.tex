% === histogramaDeAnillosConcentricos === %
\subsection{proc. histogramaDeAnillosConcentricos}
    \subsubsection{Especificaci\'on:}    
        \begin{proc}{histogramaDeAnillosConcentricos}{\In th: $eph_{h}$, \In centro: \ent$\times$\ent, \In distancias: \TLista{\ent}, \Out res: \TLista{\ent}}{}
            \pre{esTablaDeHogaresValida(th)\ \y\ esCentroValido(centro)\ \y\ sonDistanciasValidas(distancias)} 
            \post{\\
            |res| = |distancias|\ \yLuego (\\
            \tab res[0] = \#HogaresEnAnillo(th,\ centro,\ 0,\ distancias[0])\ \y\\
            \tab (\forall i:\ent)(0 < i < |result|\ \implicaLuego\\ 
                \tab\tab result[i] = \#HogaresEnAnillo(th,\ centro,\ distancias[i - 1],\ distancias[i])\\
            \tab)\\
            )} 
        \end{proc}
    
    \subsubsection{Predicados y funciones auxiliares:}
        \noindent\pred{esCentroValido}{centro : \ent$\times$\ent}{\\
        \tab -90 \leq centro_{0} \leq 90\ \y\ -180 \leq centro_{1} \leq 180\\
        }
        $\newline$
        \noindent\pred{sonDistanciasValidas}{s: \TLista{\ent}}{\\
            \tab|s| > 0\ \yLuego\ (s[0] > 0\ \y\ (\forall i:\ent)(0 \leq i < |s| - 1\ \implicaLuego\ s[i] < s[i + 1]))\\
        }
        $\newline$
        \aux{$\#$HogaresEnAnillo}{th :  $eph_{h}$, centro: \ent$\times$\ent, desde: \ent, hasta: \ent}{\ent}{\\[2ex]
            \tab\displaystyle\sum_{h \in th}\IfThenElse{cuadrado(desde) \leq distancia^{2}(h,\ centro) < cuadrado(hasta)}{1}{0}
        }
        \vspace*{1.5ex}
        $\newline$
        \aux{cuadrado}{n : \ent}{\ent}{n * n}
        $\newline$
        \aux{distancia$^{2}$}{h: $hogar$, centro: \ent$\times$\ent}{\ent}{\\[2ex]
            \tab cuadrado(h[@hoglatitud] - centro_{0}) + cuadrado(h[@hoglongitud] - centro_{1})
        }
    \vspace*{1.5ex}
    \subsubsection{Observaciones:}
        \begin{itemize}
            \item se hace uso del predicado $esTablaDeHogaresValida$ definido en 1.1.2.
            %\item el predicado $esCentroValido$ sigue con nuestra consideración de lo que constituyen una latitud y longitud válidas.
            \item dado que la pertenencia de un punto P = (x, y) a un anillo concéntrico definido en el intervalo 
            [A, B), donde A y B denotan dos radios respecto al centro C = (x$_{0}$, y$_{0}$), se define como: 
                \begin{equation}
                    A\ \leq\ \sqrt{(x - x_{0})^{2} + (y - y_{0})^{2}}\ <\ B
                \end{equation}
                por simple manipulación algebráica (elevando al cuadrado), la misma relación se mantiene para:
                \begin{equation}
                    A^{2}\ \leq\ (x - x_{0})^{2} + (y - y_{0})^{2}\ <\ B^{2}
                \end{equation}
                el predicado $\#HogaresEnAnillo$ hace uso de esta observaci\'on.
            \item cabe aclarar que el predicado $distancia^{2}$ devuelve necesariamente un entero, ya que \ent\ es un cuerpo respecto a
            la suma, resta y producto. 
        \end{itemize}