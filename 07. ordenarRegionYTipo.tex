
% === ordenarRegionYTipo === %

\subsection{proc. ordenarRegionYTipo}
    \subsubsection{Especificaci\'on:}    
        \begin{proc}{ordenarRegionYTipo}{\Inout th: $eph_{h}$, \Inout ti : $eph_{i}$}{}
            \pre{sonTablasValidas(th,\ ti)\ \y\ th = th_{0}\ \y\ ti = ti_{0}}
            \post{lasTablasNoCambian(th,\ th_{0},\ ti,\ ti_{0})\ \y\ sonTablasOrdenadas(th,\ ti)}
        \end{proc}
        % === Requisitos === %
        % === 1. Largo y elementos de th/ti == largo y elementos de th0/ti0
        % === 2. La tabla de hogares esté ordenada
        % === 2.1. Por código de región
        % === 2.2. Dentro de cada región, por CODUSU en forma creciente. (datos en tabla son enteros)
        % === 3. La tabla de individuos esté ordenada
        % === 3.1. Por mismo orden de CODUSU que la tabla de hogares ordenada. (CODUSU por región, creciente).
        % === 3.2. COMPONENTE de menor a mayor dentro de mismo hogar.
    \subsubsection{Predicados y funciones auxiliares:}
        \noindent\pred{lasTablasNoCambian}{th: $eph_{h}$,\ $th_{0}$: $eph_{h}$,\ ti: $eph_{i}$,\ $ti_{0}$: $eph_{i}$}{\\
            \tab tienenLosMismosElementos(th,\ th_{0})\ \y\ tienenLosMismosElementos(ti,\ ti_{0})\\
        }
        %$\newline$
        %\pred{tablaHogaresNoCambia}{th: $eph_{h}$, $th_{0}$: $eph_{h}$}{\\
        %    \tab|th| = |th_{0}|\ \y\\
        %    \tab(\forall h: hogar)(h \in th_{0}\ \Iff\ h \in th)\\
        %}
        %$\newline$
        %\pred{tablaIndividuosNoCambia}{ti: $eph_{i}$, $ti_{0}$: $eph_{i}$}{\\
        %    \tab|ti| = |ti_{0}|\ \y\\
        %    \tab(\forall i: individuo)(i \in ti_{0}\ \Iff\ i \in ti)\\
        %}
        $\newline$
        \pred{sonTablasOrdenadas}{th: $eph_{h}$, ti: $eph_{i}$}{\\
            \tab hogaresOrdenados(th)\ \y\ individuosOrdenados(th,ti)\\
        }
        $\newline$
        \pred{hogaresOrdenados}{th: $eph_{h}$}{\\
            \tab regionCreciente(th)\ \y\ codusuCreciente(th)\\
        }
        $\newline$
        \pred{regionCreciente}{th: $eph_{h}$}{\\
            \tab(\forall i: \ent)(0 \leq i < |th| - 1\ \implicaLuego\ th[i][@region] \leq th[i+1][@region])\\
        }
        $\newline$
        \pred{codusuCreciente}{th: $eph_{h}$}{\\
            \tab(\forall i: \ent)((0 \leq i < |th| - 1\ \yLuego\ th[i][@region] = th[i+1][@region])\ \implicaLuego\\
            \tab\tab th[i][@hogcodusu] < th[i+1][@hogcodusu]\\
            \tab)\\
        }
        $\newline$
        \pred{individuosOrdenados}{th: $eph_{h}$, ti: $eph_{i}$}{\\
           \tab codusuComoHogares(th,\ ti)\ \y\ componenteCreciente(ti)\\
        }
        $\newline$
        \pred{codusuComoHogares}{th: $eph_{h}$, ti: $eph_{i}$}{\\
            \tab(\forall i: \ent)(0 \leq i < |th| - 1\ \implicaLuego\\ 
            \tab\tab ordenadosDeADosCodusu(ti,\ th[i][@hogcodusu],\ th[i+1][@hogcodusu])\\
            \tab)\\
        }
        $\newline$
        \noindent\pred{ordenadosDeADosCodusu}{ti: $eph_{i}$, cod1: $dato$, cod2: $dato$}{\\
            % === cod1 < cod2
            \tab(\forall i,j: \ent)((0 \leq i, j < |ti|\ \yLuego\\
            \tab\tab(ti[i][@indcodusu] = cod1\ \y 
            ti[j][@indcodusu] = cod2))\ \implica\\
            \tab\tab\tab i < j\\
            \tab)\\
        }
        % === Acá es irrelevante expresar que i distinto de j; son indices de individuos con distintos codusu
        $\newline$
        \pred{componenteCreciente}{ti: $eph_{i}$}{\\
            \tab(\forall i: \ent)(0 \leq i < |ti| - 1\ \yLuego\ ti[i][@indcodusu] = ti[i+1][@indcodusu])\ \implicaLuego\\
            \tab\tab ti[i][@componente] < ti[i+1][componente]\\
            \tab)\\
        }
    \subsubsection{Observaciones:}
        \begin{itemize}
            \item se hace uso del predicado $tienenLosMismosElementos$ definido en 2.1.
            
            \item el predicado $ordenadosDeADosCodusu$ está armado para evaluar correctamente los casos en que $cod1 < cod2$. Estos
            casos se dan cuando el predicado $hogaresOrdenados$ es cierto.  
            Consideramos que esta salvedad no afecta a la evaluación final,
            ya que la verdad de $sonTablasOrdenadas$ depende de la conjunción de $hogaresOrdenados$ con $individuosOrdenados$. 
            \item Obeservamos también, sobre el anterior predicado, que:
                \begin{itemize}
                    \item el mismo verifica que todo individuo en $ti$ con el $cod1$ tiene 
                    su indice menor al de todos los individuos en $ti$ con el $cod2$.
                    \item En particular, el caso $i = j$ resulta en la condición suficiente de la implicación siendo falsa. 
                    Dado que, necesariamente, $cod1 \neq cod2$.
                \end{itemize}
            %\item el predicado $ordenadosDeADosCodusu$ considera dos CODUSU, con $cod1 < cod2$, y verifica que 
            %todo individuo en $ti$ con el $cod1$ tiene su indice menor al de todos los individuos en $ti$ con el $cod2$.
            %    \begin{itemize}
            %        \item no se puede dar el caso en el que $i = j$, ya que son \'indices de individuos con distintos CODUSU.
            %    \end{itemize}

            \item el predicado $codusuComoHogares$ comprueba si $ordenadosDeADosCodusu$ es verdadero o no para todos los 
            codusu de a pares de hogares consecutivos en $th$. Considera que la tabla de hogares está ordenada por regi\'on 
            y codusu creciente.
            %\item el predicado $codusuComoHogares$ comprueba si $ordenadosDeADosCodusu$ es verdadero o no para todos los 
            %CODUSU de pares de hogares consecutivos en $th$. Considera la tabla de hogares ya ordenada por regi\'on y CODUSU creciente.
            
            \item ambos predicados, $codusuComoHogares$ y $ordenadosDeADoscodusu$, funcionan en conjunto para verificar que, 
            en la tabla de individuos ordenada, todos los individuos est\'en agrupados por codusu, 
            y que estos sigan el mismo orden que el de los hogcodusu de la tabla de hogares ya ordenada.
            %\item ambos predicados, $codusuComoHogares$ y $ordenadosDeADosCodusu$, funcionan en conjunto para verificar que, 
            %en la tabla de individuos ordenada, todos los individuos est\'en agrupados por mismo CODUSU, 
            %y que estos sigan el orden que el de los HOGCODUSU de la tabla de hogares ya ordenada.
        \end{itemize}
