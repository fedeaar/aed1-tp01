% === corregirRegion === %
\subsection{proc. corregirRegion}

    \subsubsection{Especificaci\'on:}  

    \begin{proc}{corregirRegion}{\Inout th : $eph_{h}$, \In ti : $eph_{i}$}{}
            \pre{sonTablasValidas(th,\ ti)\ \y\ th = th_{0}}
            \post{|th| = |th_0|\ \yLuego\ (fueraDeGBAPermaneceSinCambios(th,\ th_{0})\ \y\ cambi$\'o$GBAaPampeana(th,\ th_{0}))}
            % creo que faltaria chequear que todos los elementos que no tienen region 1 son iguales a los de th0
            % tambien algo del estilo |th| = |th0|
            % RESPUESTA: hubo enroque de predicados "nohayregion1" por el "fueradegba..." Tambien incorporé la igualdad de largos.
           
        \end{proc}
    
    \subsubsection{Predicados y funciones auxiliares:}
    % obs: como en el anterior, le falta el tipo a los argumentos de algunos pred
    
    \noindent\pred{fueraDeGBAPermaneceSinCambios}{th, $th_{0}$ : $eph_{h}$}{\\
        \tab (\forall i:\ent)((0 \leq i < |th|\ \y\ th_{0}[i][@Region] \neq 1)\ \implicaLuego\ th[i] = th_{0}[i])\\
    }       
    $\newline$
    \noindent\pred{cambióGBAaPampeana}{th, th$_{0}$ : $eph_{h}$}{\\
        % corrijo: el implica en este caso tiene que ser implicaluego
        \tab (\forall i:\ent)((0 \leq i < |th|\ \y\ th_{0}[i][@Region] = 1)\ \implicaLuego\ estaEnRegionPampeana(th[i],\ th_{0}[i]))\\
    }
    $\newline$
    \noindent\pred{estaEnRegionPampeana}{h , h$_{0}$ : $hogar$}{\\
        % corrijo: el implica en este caso tiene que ser implicaluego
        % cambiaria pasarle el indice por pasarle el hogar y usar eso
        % RESPUESTA: hecho tal cual se sugirió
        \tab (\forall j:\ent)((0 \leq j < |h|\ \y\ j \neq @Region)\ \implicaLuego\ h[j] = h_{0}[j])\ \y\ h[@Region] = 5\\
    }

    \subsubsection{Observaciones:}
        \begin{itemize}
            \item las condiciones suficientes en los predicados $fueraDeGBAPermaneceSinCambios$ y $cambi$\'o$GBAaPampeana$
            son complementarias, por lo que todo hogar que pertenece a th$_{0}$, pertenece a th (dada la salvedad que cambian las regiones).
            Y, dado que tienen el mismo largo, no puede haber hogares nuevos.
        \end{itemize}
