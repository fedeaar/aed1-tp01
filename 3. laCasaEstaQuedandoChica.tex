% === laCasaEstaQuedandoChica === % 
\subsection{proc. laCasaEstaQuedandoChica}

    \subsubsection{Especificaci\'on:}
        \begin{proc}{laCasaEstaQuedandoChica}{\In th: $eph_{h}$, \In ti: $eph_{i}$, \Out res: \TLista{\float}}{}
            \pre{sonEncuestasValidas(th,\ ti)\ \y\ 1 \leq region \leq 6}
            \post{|res| = 6\ \yLuego\ (\forall region: dato)(1 \leq region \leq 6\ \implicaLuego\ res[region - 1] = \%hacinado(th,\ ti,\ region))}
        \end{proc}

    \subsubsection{Predicados y funciones auxiliares:}
        \noindent\pred{$\Omega$NoVacio$_{1{.}3}$}{th: $eph_{h}$, region: $dato$}{\\
            \tab(\exists h: $hogar$)(h \in th\ \yLuego\ esHogarValido_{1{.}3}(h,\ region))\\
        }
        $\newline$
        \noindent\pred{esHogarValido$_{1{.}3}$}{h: $hogar$, region: $dato$}{\\
            \tab h[@region] = region\ \y\ h[@mas\_500] = 0\ \y\ h[@iv1] = 1\\
        }
        $\newline$
        \noindent\pred{casaHacinada}{ti: $eph_{i}$, h: $hogar$, region: $dato$}{\\
            \tab esHogarValido_{1{.}3}(h, region)\ \y\ \#individuosEnHogar(ti,\ h[@hogcodusu]) > 3 * h[@iv2]\\
        }
        $\newline$
        \noindent\aux{$\%$hacinado}{th: $eph_{h}$, ti: $eph_{i}$, region: $dato$}{\float}{\\[2ex]
            \tab\IfThenElse{{\Omega}NoVacio_{1{.}3}(th,\ region)}{
                \frac{\displaystyle\sum_{h \in th}{(\IfThenElse{casaHacinada(ti,\ h,\ region)}{1}{0})}}
                    {\displaystyle\sum_{h \in th}{(\IfThenElse{esHogarValido_{1{.}3}(h,\ region)}{1}{0})}}
            }{0}
        }
    
    \subsubsection{Observaciones:}
        \begin{itemize}
            \item se hace uso de la funci\'on auxiliar ${\#}individuosEnHogar$ definida en 2.2.
            \item la funci\'on auxiliar ${\%}hacinado$ considera como espacio de probabilidad ($\Omega$) a todos los hogares que cumplan 
            con el predicado $esHogarValido_{1{.}3}$.
            \item en el predicado $\%hacinado$ consideramos que si no hay hogares v\'alidos en una región, entonces la proporción de hogares hacinados respecto a esa región es 0.
        \end{itemize}