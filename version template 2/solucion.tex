% === TODO ===
% poner en decisiones tomadas porque decidimos considerar el caso del universo vacio en el 4.;
% aux con #
% decisiones tomadas 1810
% pred max habitaciones que tenga relacion a th
% 1.4. declaratividad
% aux general matriz declaratividad
% indicehogarporcodusu poner las condiciones en el aux
% pasar a decisiones tomadas lo de v3.

\documentclass[a4paper]{article} 

\setlength{\parskip}{0.1em}
\newcommand{\tab}[1][1.2cm]{\hspace*{#1}}
\input{Algo1Macros}
\usepackage{caratula} % Version modificada para usar las macros de algo1 de ~> https://github.com/bcardiff/dc-tex
\usepackage{xcolor}

% === macros === %
\newcommand{\type}[2]{%
    {\normalfont\bfseries\ttfamily\noindent type\ }%
    {\normalfont\ttfamily #1}
    {= #2\\}%
}
\newcommand{\enum}[2]{%
    {\normalfont\bfseries\ttfamily\noindent enum\ }%
    {\normalfont\ttfamily #1}
    {\{\\\tab#2\\\}}%
}
\newcommand{\comment}[2]{%
    \\[0.5ex]
    {#1\comentario{\text{#2}}}
    \\[0.5ex]
}


\begin{document}

% === caratula === %
\titulo{TP de Especificaci\'on}
\subtitulo{An\'alisis Habitacional Argentino}
\fecha{8 de Septiembre de 2021}
\materia{Algoritmos y Estructuras de Datos I}
\grupo{Grupo 02, comisi\'on 11}
\newcommand{\senial}{\textit{se\~nal}}

\integrante{Lakowsky, Manuel}{511/21}{mlakowsky@gmail.com}
\integrante{Lenardi, Juan Manuel}{56/14}{juanlenardi@gmail.com}
\integrante{Arienti, Federico}{316/21}{fa.arienti@gmail.com}

\maketitle

% === problemas === %
\section{Problemas}

% === esEncuestaValida === %
\subsection{proc. esEncuestaValida}

    \begin{proc}{esEncuestaValida}{\In th: $eph_{h}$, \In ti : $eph_{i}$, \Out result: \bool}{}
        \pre{\True}
        \post{res = \True \Iff validarEncuesta(th,\ ti)}
    \end{proc}

    $\newline$
    \noindent\pred{validarEncuesta}{th: $eph_{h}$, ti: $eph_{i}$}{
        \comment{\tab}{tabla hogares}
        \tab (esTabla(th,\ @largoItemHogar)\ \yLuego\\
        \tab (\forall h: hogar)(h \in th\ \implicaLuego\ (\\
        \tab\tab codigoValido_h(th,\ ti,\ h)\ \y\ a\tilde{n}oyTrimestreCongruente_h(th,\ h)\ \y\ attEnRango_h(h)\\
        \tab))) \y
        \comment{\tab}{tabla individuos}
        \tab (esTabla(ti,\ @largoItemIndividuo)\ \yLuego\\
        \tab(\forall i: individuo)(i \in ti\ \implicaLuego\ (\\
        \tab\tab codigoValido_i(th,\ ti,\ i)\ \y\ a\tilde{n}oyTrimestreCongruente_i(th,\ i)\ \y\ attEnRango_i(i)\ \y\\
        \tab\tab validarComponente_i(ti,\ i)\\
        \tab)))\\
    }

    $\newline$
    \pred{codigoValido$_{h}$}{th: $eph_{h}$, ti: $eph_{i}$, h: $hogar$}{\\
        \tab(\exists i: individuo)(i \in ti\ \yLuego\\
        \tab\tab h[@hogcodusu] = i[@indcodusu]\\
        \tab) \y\\
        \tab\neg(\exists h_2: hogar)((h_2 \in th\ \y\ h_2 \neq h)\ \yLuego\\ 
        \tab\tab h[@hogcodusu] = h_2[@hogcodusu]\\
        \tab)\\
    }

    $\newline$
    \pred{añoyTrimestreCongruente$_{h}$}{th: $eph_{h}$, h: $hogar$}{\\
        \tab h[@hoga\tilde{n}o] = th[0][@hoga\tilde{n}o]\ \y \ h[@hogtrimestre] = th[0][@hogtrimestre]\\
    }

    $\newline$
    \pred{attEnRango$_{h}$}{h: $hogar$}{\\
        \tab 0 \leq h[@hogcodusu]\ \y\\ 
        \tab 1810 \leq h[@hoga\tilde{n}o]\ \y\\
        \tab 1 \leq h[@hogtrimestre] \leq 4\ \y\\
        \tab -90 \leq th[i][@hoglatitud] \leq 90\ \y\\
        \tab -180 \leq th[i][@hoglongitud] \leq 180\ \y\\
        \tab 1 \leq h[@ii7] \leq 3\ \y\\ 
        \tab 1 \leq h[@region] \leq 6\ \y\\ 
        \tab 0 \leq h[@mas\_500] \leq 1\ \y\\
        \tab 1 \leq h[@iv1] \leq 5\ \y\\ 
        \tab 0 < h[@ii2] \leq h[@iv2]\ \y\\
        \tab 1 \leq h[@ii3] \leq 2\\
    }

    $\newline$          
    \noindent\pred{codigoValido$_{i}$}{th: $eph_{h}$, ti: $eph_{i}$, i: $individuo$}{\\
        \tab(\exists h: hogar)(h \in th\ \yLuego \\
        \tab\tab i[@indcodusu] = h[@hogcodusu]\\
        \tab) \y\\
        \tab\neg(\exists i_2: individuo)((i_2 \in ti\ \y\  i_2 \neq i)\ \yLuego\ (\\
        \tab\tab i[@indcodusu] = i_2[@indcodusu]\ \y\ i[@componente] = i_2[@componente]\\
        \tab))\\
    }

    $\newline$
    \pred{añoyTrimestreCongruente$_{i}$}{th: $eph_{h}$, i: $individuo$}{\\
        \tab i[@inda\tilde{n}o] = th[0][@hoga\tilde{n}o]\ \y\ i[@indtrimestre] = th[0][@hogtrimestre]\\
    }

    $\newline$
    \pred{attEnRango$_{i}$}{i: $individuo$}{\\
        \tab 0 \leq i[@indcodusu]\ \y\\ 
        \tab 1 \leq i[@componente] <= 20\ \y\\
        \tab 1 \leq i[@ch4] \leq 2\ \y\\
        \tab 0 \leq i[@ch6]\ \y\\
        \tab 0 \leq i[@nivel\_ed] \leq 1\ \y\\ 
        \tab -1 \leq i[@estado] \leq 1\ \y\\ 
        \tab 0 \leq i[@cat\_ocup] \leq 4\ \y\\ 
        \tab -1 \leq i[@p47t]\ \y\\
        \tab 1 \leq i[@ppo4g] \leq 10\\    
    }

    $\newline$
    \pred{validarComponente$_{i}$}{ti: $eph_{i}$, i: $individuo$}{\\
        \tab i[@componente] = 1\ \vee (\exists i_2: individuo)(i_2 \in ti\ \yLuego\ i[@componente] - 1 = i_2[@componente])\\
    }

% === histHabitacional === %
$\newline$   
\subsection{proc. histHabitacional}

    \begin{proc}{histHabitacional}{\In th: $eph_{h}$, \In ti: $eph_{i}$, \In region: \ent, \Out res: \TLista{\ent}}{}
    \pre{validarEncuesta(th,\ ti)\ \y\ HayCasasEnLaRegion(th,\ region)}
    % CORREGIR
    \post{\textcolor{red}{HayMaximoDeHabitaciones(res)}\ \yLuego \\
    \hspace*{0.5cm} (\forall i:\ent)(0\leq i < |res|\ \implicaLuego\ res[i] = \#casasPorNroDeHabitaciones(th,\ k,\ i + 1))\\
    }
    \end{proc}

    $\newline$
    \noindent\pred{HayCasasEnLaRegion}{th: $eph_{h}$, region: \ent}{\\
        \tab(\exists h: hogar)(h \in th\ \yLuego\ esHogarValidoParaHist(h,\ region))\\
    }

    $\newline$
    \noindent\pred{esHogarValidoParaHist}{h: $hogar$, region: \ent}{\\
        \tab h[@region] = region\ \y\ h[@iv1] = 1\\
    }

    $\newline$
    % CORREGIR
    \textcolor{red}{
    \noindent\pred{HayMaximoDeHabitaciones}{res: \TLista{\ent}}{\\
        \tab(\exists max: \ent)(max = |res|\ \y\ res[max-1] > 0)\\
    }
    }

    $\newline$
    \noindent\aux{$\#$casasPorNroDeHabitaciones}{th: $eph_{h}$, region: \ent, habitaciones: \ent}{\ent}{\\[0.5ex]
        \tab\displaystyle\sum_{h \in th}
        {(\IfThenElse {esHogarValidoParaHist(h,\ region)\ \y\ h[@iv2] = habitaciones}{1}{0})}
    }

% === laCasaEstaQuedandoChica === %
$\newline$   
\subsection{proc. laCasaEstaQuedandoChica}

    \begin{proc}{laCasaEstaQuedandoChica}{\In th: $eph_{h}$, \In ti: $eph_{i}$, \Out res: \TLista{\ent}}{}
    \pre{validarEncuesta(th,\ ti)}
    \post{|res| = 6\ \yLuego\ (\forall region: \ent)(1 \leq region \leq 6\ \implicaLuego\ res[region - 1] = \%hacinado(th,\ ti,\ region)}
    \end{proc}

    $\newline$
    \noindent\aux{$\%$hacinado}{th: $eph_{h}$, ti: $eph_{i}$, region: \ent}{\float}{\\
        \tab\IfThenElse{{\Omega}hacinadosNoVacio(th,\ region)}{
            \frac{\displaystyle\sum_{h \in th}{(\IfThenElse{casaHacinada(ti,\ h,\ region)}{1}{0})}}
                 {\displaystyle\sum_{h \in th}{(\IfThenElse{esHogarValidoParasHacinamiento(h,\ region)}{1}{0})}}
        }{0}
    }
    $\newline$
    \noindent\pred{$\Omega$hacinadosNoVacio}{th: $eph_{h}$, region: \ent}{\\
        \tab(\exists h: $hogar$)(h \in th\ \yLuego\ esHogarValidoParaHacinamiento(h,\ region))\\
    }

    $\newline$
    \noindent\pred{esHogarValidoParaHacinamiento}{h: $hogar$, region: \ent}{\\
        \tab h[@region] = region\ \y\ h[@mas500] = 0\ \y\ h[@iv1] = 1\\
    }

    $\newline$
    \noindent\pred{casaHacinada}{ti: $eph_{i}$, h: $hogar$, region: \ent}{\\
        \tab esHogarValidoParaHacinamiento(h, region)\ \y\ \#individuosEnHogar(ti,\ h[@hogcodusu]) > 3 * h[@iv2]\\
    }

% === creceElTeleworkingEnCiudadesGrandes === %
$\newline$   
\subsection{proc. creceElTeleworkingEnCiudadesGrandes}

    \begin{proc}{creceElTeleworkingEnCiudadesGrandes}{\In t1h: $eph_{h}$, \In t1i: $eph_{i}$, \In t2h: $eph_{h}$, \In t2i: $eph_{i}$, \Out res: \bool}{}
        \pre{\\
            (validarEncuesta(t1h,\ t1i)\ \y\ validarEncuesta(t2h,\ t2i)) \yLuego\\
            (t1h[0][@hoga\tilde{n}o] = t2h[0][@hoga\tilde{n}o] - 1\ \y\
            % \tab t1h[0][@hoga\tilde{n}o] < t2h[0][@hoga\tilde{n}o] \y \\
            t1h[0][@hogtrimestre] = t2h[0][@hogtrimestre])\\
        }
        \post{res = \True \iff $\%$teleworking(t1h,\ t1i) < $\%$teleworking(t2h,\ t2i)}
    \end{proc}

    $\newline$
    \aux{$\%$teleworking}{th: $eph_{h}$, ti: $eph_{i}$}{\float}{\\
        \tab\IfThenElse{{\Omega}teleworkingNoVacio(th)}{
            \frac{
                \displaystyle\sum_{i \in ti}(\IfThenElse{haceTeleworking(th,\ i)}{1}{0})
            }{
                \displaystyle\sum_{i \in ti}(\IfThenElse{viveEnHogarValido(th,\ i)}{1}{0})
            }
        }{0}
    }

    $\newline$
    \pred{$\Omega$teleworkingNoVacio}{th: $eph_{h}$}{\\
        \tab(\exists h: $hogar$)(h \in th\ \yLuego\ esHogarValidoParaTeleworking(h))\\
    }

    $\newline$
    \pred{esHogarValidoParaTeleworking}{h: $hogar$}{\\
        \comment{\tab}{Hogar cumple con especificaciones}
        \tab h[@mas\_500] = 1\ \y\ (h[@iv1] = 1\ \vee\ h[@iv1] = 2)\\
    }

    $\newline$
    \pred{haceTeleworking}{th: $eph_{h}$, i: $individuo$}{\\
        \comment{\tab}{Hogar e Individuo cumplen con especificaciones}
        \tab viveEnHogarValido(th, i)\ \y\ i[@ii3] = 1\ \y\ i[@ppo4g] = 6\\
    }

    $\newline$
    \pred{viveEnHogarValido}{th: $eph_{h}$, i: $individuo$}{\\
        \comment{\tab}{Hogar del individuo cumple con especificaciones}
        \tab esHogarValidoParaTeleworking(th[indiceHogarPorCodusu(th,\ i[@indcodusu])])\\
    }

% === costoSubsidioMejora === %
$\newline$   
\subsection{proc. costoSubsidioMejora}

    \begin{proc}{costoSubsidioMejora}{\In th: $eph_{i}$, \In ti: $eph_{i}$, \In monto: \ent, \Out res: \ent}{}
        \pre{validarEncuesta(th,\ ti) \y monto > 0} 
        \post{res = monto * \displaystyle\sum_{h \in th}(\IfThenElse{esHogarValidoParaSubsidio(ti,\ h)}{1}{0})}    
    \end{proc}

    $\newline$   
    \pred{esHogarValidoParaSubsidio}{ti: $eph_{i}$, h: $hogar$}{\\
        \tab h[@ii7] = 1\ \y\ h[@iv1] = 1\ \y\ individuosEnHogar(ti,\ h[@hogcodusu]) - 2 > h[@ii2]\\
    }

% === generales === %
\section{Predicados y Auxiliares generales}
                    
\subsection{predicados generales}
        
    \noindent\pred{esMatriz}{s: \TLista{\TLista{T}}}{\\
        \tab(\forall fila: \TLista{T})(fila \in s\ \implicaLuego\ |fila| = |s[0]|)\\
    }
    $\newline$
    \noindent\pred{esTabla}{m: \TLista{\TLista{T}}, columnas: \ent}{\\
        \tab |m| > 0 \yLuego (|m[0]| = columnas \y esMatriz(m))\\
    }

\subsection{auxiliares generales}     
    
    \noindent\aux{$\#$individuosEnHogar}{ti: $eph_{i}$, codusu$_{h}$: \ent}{\ent}{
        \displaystyle\sum_{i \in ti}(\IfThenElse {i[@indcodusu] = codusu_h}{1}{0})
    }
    
    $\comment{}{indiceHogarPorCodusu asume codusu$_{h}$ existe en la tabla y es único}$
    \noindent\aux{indiceHogarPorCodusu}{th: $eph_{h}$, codusu$_{h}$: \ent}{\ent}{
        \displaystyle\sum_{h \in th}\IfThenElse{h[@hogcodusu] = codusu_h}{i}{0}
    }

\subsection{tipos y enumerados}
    \type {dato}{\ent}
    \type {individuo}{\TLista{dato}} 
    \type {hogar}{\TLista{dato}}
    \type {eph$_i$}{\TLista{individuo}}
    \type {eph$_h$}{\TLista{hogar}}
    \type {joinHI}{\TLista{hogar \times individuo}}

    \enum {ItemHogar}{hogcodusu, hogaño, hogtrimestre, hoglatitud, hoglongitud, ii7, region, mas\_500, iv1, iv2, ii2, ii3} 
    \\
    \enum {ItemIndividuo}{indcodusu, componente, indaño, indtrimestre, ch4, ch6, nivel\_ed, cat\_ocup, p47t, ppo4g}

\subsection{referencias}
    \noindent\aux{@hogcodusu}{}{\ent}{itemHogar.ord(hogcodusu)}
    \noindent\aux{@hogaño}{}{\ent}{itemHogar.ord(hoga\tilde{n}o)}
    \noindent\aux{@hogtrimestre}{}{\ent}{itemHogar.ord(hogtrimestre)}
    \noindent\aux{@hoglatitud}{}{\ent}{itemHogar.ord(hoglatitud)}
    \noindent\aux{@hoglongitud}{}{\ent}{itemHogar.ord(hoglongitud)}
    \noindent\aux{@ii7}{}{\ent}{itemHogar.ord(ii7)}
    \noindent\aux{@region}{}{\ent}{itemHogar.ord(region)}
    \noindent\aux{@mas\_500}{}{\ent}{itemHogar.ord(mas\_500)}
    \noindent\aux{@iv1}{}{\ent}{itemHogar.ord(iv1)}
    \noindent\aux{@iv2}{}{\ent}{itemHogar.ord(iv2)}
    \noindent\aux{@ii2}{}{\ent}{itemHogar.ord(ii2)}
    \noindent\aux{@ii3}{}{\ent}{itemHogar.ord(ii3)}
    $\newline$
    \noindent\aux{@indcodusu}{}{\ent}{itemIndividuo.ord(indcodusu)}
    \noindent\aux{@componente}{}{\ent}{itemIndividuo.ord(componente)}
    \noindent\aux{@indaño}{}{\ent}{itemIndividuo.ord(inda\tilde{n}o)}
    \noindent\aux{@indtrimestre}{}{\ent}{itemIndividuo.ord(indtrimestre)}
    \noindent\aux{@ch4}{}{\ent}{itemIndividuo.ord(ch4)}
    \noindent\aux{@ch6}{}{\ent}{itemIndividuo.ord(ch6)}
    \noindent\aux{@nivel\_ed}{}{\ent}{itemIndividuo.ord(nivel\_ed)}
    \noindent\aux{@cat\_ocup}{}{\ent}{itemIndividuo.ord(cat\_ocup)}
    \noindent\aux{@p47t}{}{\ent}{itemIndividuo.ord(p47t)}
    \noindent\aux{@ppo4g}{}{\ent}{itemIndividuo.ord(ppo4g)}
    $\newline$
    \noindent\aux{@largoItemHogar}{}{\ent}{12}
    \noindent\aux{@largoitemIndividuo}{}{\ent}{10}

\section{Decisiones tomadas}
\subsection{Ejercicio 4.2. Igualdad res[i]=HabitacionesPorCasa(th,region,i+1) en Postcondicion}
    \noindent {Se asume que de existir al menos una casa entonces la misma posee al menos una habitacion o un ambiente usado a tal fin. No hay casa con cero habitaciones}

\end{document}

