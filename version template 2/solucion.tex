% === TODO ===
% pred max habitaciones que tenga relacion a th

\documentclass[a4paper]{article} 

\setlength{\parskip}{0.1em}
\newcommand{\tab}[1][1.2cm]{\hspace*{#1}}
\usepackage{ifthen}
\usepackage{amssymb}
\usepackage{multicol}
\usepackage{graphicx}
\usepackage[absolute]{textpos}
\usepackage{amsmath, amscd, amssymb, amsthm, latexsym}
\usepackage[noload]{qtree}
%\usepackage{xspace,rotating,calligra,dsfont,ifthen}
\usepackage{xspace,rotating,dsfont,ifthen}
\usepackage[spanish,activeacute]{babel}
\usepackage[utf8]{inputenc}
\usepackage{pgfpages}
\usepackage{pgf,pgfarrows,pgfnodes,pgfautomata,pgfheaps,xspace,dsfont}
\usepackage{listings}
\usepackage{multicol}
\usepackage{todonotes}
\usepackage{url}
\usepackage{float}
\usepackage{framed}


\makeatletter

\@ifclassloaded{beamer}{%
  \newcommand{\tocarEspacios}{%
    \addtolength{\leftskip}{4em}%
    \addtolength{\parindent}{-3em}%
  }%
}
{%
  \usepackage[top=1cm,bottom=2cm,left=1cm,right=1cm]{geometry}%
  \usepackage{color}%
  \newcommand{\tocarEspacios}{%
    \addtolength{\leftskip}{5em}%
    \addtolength{\parindent}{-3em}%
  }%
}

\newcommand{\encabezadoDeProc}[4]{%
  % Ponemos la palabrita problema en tt
%  \noindent%
  {\normalfont\bfseries\ttfamily proc}%
  % Ponemos el nombre del problema
  \ %
  {\normalfont\ttfamily #2}%
  \
  % Ponemos los parametros
  (#3)%
  \ifthenelse{\equal{#4}{}}{}{%
  \ =\ %
  % Ponemos el nombre del resultado
  {\normalfont\ttfamily #1}%
  % Por ultimo, va el tipo del resultado
  \ : #4}
}

\newcommand{\encabezadoDeTipo}[2]{%
  % Ponemos la palabrita tipo en tt
  {\normalfont\bfseries\ttfamily tipo}%
  % Ponemos el nombre del tipo
  \ %
  {\normalfont\ttfamily #2}%
  \ifthenelse{\equal{#1}{}}{}{$\langle$#1$\rangle$}
}

% Primero definiciones de cosas al estilo title, author, date

\def\materia#1{\gdef\@materia{#1}}
\def\@materia{No especifi\'o la materia}
\def\lamateria{\@materia}

\def\cuatrimestre#1{\gdef\@cuatrimestre{#1}}
\def\@cuatrimestre{No especifi\'o el cuatrimestre}
\def\elcuatrimestre{\@cuatrimestre}

\def\anio#1{\gdef\@anio{#1}}
\def\@anio{No especifi\'o el anio}
\def\elanio{\@anio}

\def\fecha#1{\gdef\@fecha{#1}}
\def\@fecha{\today}
\def\lafecha{\@fecha}

\def\nombre#1{\gdef\@nombre{#1}}
\def\@nombre{No especific'o el nombre}
\def\elnombre{\@nombre}

\def\practicas#1{\gdef\@practica{#1}}
\def\@practica{No especifi\'o el n\'umero de pr\'actica}
\def\lapractica{\@practica}


% Esta macro convierte el numero de cuatrimestre a palabras
\newcommand{\cuatrimestreLindo}{
  \ifthenelse{\equal{\elcuatrimestre}{1}}
  {Primer cuatrimestre}
  {\ifthenelse{\equal{\elcuatrimestre}{2}}
  {Segundo cuatrimestre}
  {Verano}}
}


\newcommand{\depto}{{UBA -- Facultad de Ciencias Exactas y Naturales --
      Departamento de Computaci\'on}}

\newcommand{\titulopractica}{
  \centerline{\depto}
  \vspace{1ex}
  \centerline{{\Large\lamateria}}
  \vspace{0.5ex}
  \centerline{\cuatrimestreLindo de \elanio}
  \vspace{2ex}
  \centerline{{\huge Pr\'actica \lapractica -- \elnombre}}
  \vspace{5ex}
  \arreglarincisos
  \newcounter{ejercicio}
  \newenvironment{ejercicio}{\stepcounter{ejercicio}\textbf{Ejercicio
      \theejercicio}%
    \renewcommand\@currentlabel{\theejercicio}%
  }{\vspace{0.2cm}}
}


\newcommand{\titulotp}{
  \centerline{\depto}
  \vspace{1ex}
  \centerline{{\Large\lamateria}}
  \vspace{0.5ex}
  \centerline{\cuatrimestreLindo de \elanio}
  \vspace{0.5ex}
  \centerline{\lafecha}
  \vspace{2ex}
  \centerline{{\huge\elnombre}}
  \vspace{5ex}
}


%practicas
\newcommand{\practica}[2]{%
    \title{Pr\'actica #1 \\ #2}
    \author{Algoritmos y Estructuras de Datos I}
    \date{Primer Cuatrimestre 2019}

    \maketitlepractica{#1}{#2}
}

\newcommand \maketitlepractica[2] {%
\begin{center}
\begin{tabular}{r cr}
 \begin{tabular}{c}
{\large\bf\textsf{\ Algoritmos y Estructuras de Datos I\ }}\\
Primer Cuatrimestre 2019\\
\title{\normalsize Gu\'ia Pr\'actica #1 \\ \textbf{#2}}\\
\@title
\end{tabular} &
\begin{tabular}{@{} p{1.6cm} @{}}
\includegraphics[width=1.6cm]{logodpt.jpg}
\end{tabular} &
\begin{tabular}{l @{}}
 \emph{Departamento de Computaci\'on} \\
 \emph{Facultad de Ciencias Exactas y Naturales} \\
 \emph{Universidad de Buenos Aires} \\
\end{tabular}
\end{tabular}
\end{center}

\bigskip
}


% Simbolos varios

\newcommand{\nat}{\ensuremath{\mathds{N}}}
\newcommand{\ent}{\ensuremath{\mathds{Z}}}
\newcommand{\float}{\ensuremath{\mathds{R}}}
\newcommand{\bool}{\ensuremath{\mathsf{Bool}}}
\newcommand{\True}{\ensuremath{\mathrm{true}}}
\newcommand{\False}{\ensuremath{\mathrm{false}}}
\newcommand{\Then}{\ensuremath{\rightarrow}}
\newcommand{\Iff}{\ensuremath{\leftrightarrow}}
\newcommand{\implica}{\ensuremath{\longrightarrow}}
\newcommand{\IfThenElse}[3]{\ensuremath{\mathsf{if}\ #1\ \mathsf{then}\ #2\ \mathsf{else}\ #3\ \mathsf{fi}}}
\newcommand{\In}{\textsf{in }}
\newcommand{\Out}{\textsf{out }}
\newcommand{\Inout}{\textsf{inout }}
\newcommand{\yLuego}{\land _L}
\newcommand{\oLuego}{\lor _L}
\newcommand{\implicaLuego}{\implica _L}
\newcommand{\existe}[3]{\ensuremath{(\exists #1:\ent) \ #2 \leq #1 < #3 \ }}
\newcommand{\paraTodo}[3]{\ensuremath{(\forall #1:\ent) \ #2 \leq #1 < #3 \ }}

% Símbolo para marcar los ejercicios importantes (estrellita)
\newcommand\importante{\raisebox{0.5pt}{\ensuremath{\bigstar}}}


\newcommand{\rango}[2]{[#1\twodots#2]}
\newcommand{\comp}[2]{[\,#1\,|\,#2\,]}

\newcommand{\rangoac}[2]{(#1\twodots#2]}
\newcommand{\rangoca}[2]{[#1\twodots#2)}
\newcommand{\rangoaa}[2]{(#1\twodots#2)}

%ejercicios
\newtheorem{exercise}{Ejercicio}
\newenvironment{ejercicio}[1][]{\begin{exercise}#1\rm}{\end{exercise} \vspace{0.2cm}}
\newenvironment{items}{\begin{enumerate}[a)]}{\end{enumerate}}
\newenvironment{subitems}{\begin{enumerate}[i)]}{\end{enumerate}}
\newcommand{\sugerencia}[1]{\noindent \textbf{Sugerencia:} #1}

\lstnewenvironment{code}{
    \lstset{% general command to set parameter(s)
        language=C++, basicstyle=\small\ttfamily, keywordstyle=\slshape,
        emph=[1]{tipo,usa}, emphstyle={[1]\sffamily\bfseries},
        morekeywords={tint,forn,forsn},
        basewidth={0.47em,0.40em},
        columns=fixed, fontadjust, resetmargins, xrightmargin=5pt, xleftmargin=15pt,
        flexiblecolumns=false, tabsize=2, breaklines, breakatwhitespace=false, extendedchars=true,
        numbers=left, numberstyle=\tiny, stepnumber=1, numbersep=9pt,
        frame=l, framesep=3pt,
    }
   \csname lst@SetFirstLabel\endcsname}
  {\csname lst@SaveFirstLabel\endcsname}


%tipos basicos
\newcommand{\rea}{\ensuremath{\mathsf{Float}}}
\newcommand{\cha}{\ensuremath{\mathsf{Char}}}
\newcommand{\str}{\ensuremath{\mathsf{String}}}

\newcommand{\mcd}{\mathrm{mcd}}
\newcommand{\prm}[1]{\ensuremath{\mathsf{prm}(#1)}}
\newcommand{\sgd}[1]{\ensuremath{\mathsf{sgd}(#1)}}

\newcommand{\tuple}[2]{\ensuremath{#1 \times #2}}

%listas
\newcommand{\TLista}[1]{\ensuremath{seq \langle #1\rangle}}
\newcommand{\lvacia}{\ensuremath{[\ ]}}
\newcommand{\lv}{\ensuremath{[\ ]}}
\newcommand{\longitud}[1]{\ensuremath{|#1|}}
\newcommand{\cons}[1]{\ensuremath{\mathsf{addFirst}}(#1)}
\newcommand{\indice}[1]{\ensuremath{\mathsf{indice}}(#1)}
\newcommand{\conc}[1]{\ensuremath{\mathsf{concat}}(#1)}
\newcommand{\cab}[1]{\ensuremath{\mathsf{head}}(#1)}
\newcommand{\cola}[1]{\ensuremath{\mathsf{tail}}(#1)}
\newcommand{\sub}[1]{\ensuremath{\mathsf{subseq}}(#1)}
\newcommand{\en}[1]{\ensuremath{\mathsf{en}}(#1)}
\newcommand{\cuenta}[2]{\mathsf{cuenta}\ensuremath{(#1, #2)}}
\newcommand{\suma}[1]{\mathsf{suma}(#1)}
\newcommand{\twodots}{\ensuremath{\mathrm{..}}}
\newcommand{\masmas}{\ensuremath{++}}
\newcommand{\matriz}[1]{\TLista{\TLista{#1}}}

% Acumulador
\newcommand{\acum}[1]{\ensuremath{\mathsf{acum}}(#1)}
\newcommand{\acumselec}[3]{\ensuremath{\mathrm{acum}(#1 |  #2, #3)}}

% \selector{variable}{dominio}
\newcommand{\selector}[2]{#1~\ensuremath{\leftarrow}~#2}
\newcommand{\selec}{\ensuremath{\leftarrow}}

\newcommand{\pred}[3]{%
    {\normalfont\bfseries\ttfamily pred }%
    {\normalfont\ttfamily #1}%
    \ifthenelse{\equal{#2}{}}{}{\ (#2) }%
    \{\ensuremath{#3}\}%
    {\normalfont\bfseries\,\par}%
  }

\newenvironment{proc}[4][res]{%
  % El parametro 1 (opcional) es el nombre del resultado
  % El parametro 2 es el nombre del problema
  % El parametro 3 son los parametros
  % El parametro 4 es el tipo del resultado
  % Preambulo del ambiente problema
  % Tenemos que definir los comandos requiere, asegura, modifica y aux
  \newcommand{\pre}[2][]{%
    {\normalfont\bfseries\ttfamily Pre}%
    \ifthenelse{\equal{##1}{}}{}{\ {\normalfont\ttfamily ##1} :}\ %
    \{\ensuremath{##2}\}%
    {\normalfont\bfseries\,\par}%
  }
  \newcommand{\post}[2][]{%
    {\normalfont\bfseries\ttfamily Post}%
    \ifthenelse{\equal{##1}{}}{}{\ {\normalfont\ttfamily ##1} :}\
    \{\ensuremath{##2}\}%
    {\normalfont\bfseries\,\par}%
  }
  \renewcommand{\aux}[4]{%
    {\normalfont\bfseries\ttfamily aux\ }%
    {\normalfont\ttfamily ##1}%
    \ifthenelse{\equal{##2}{}}{}{\ (##2)}\ : ##3\, = \ensuremath{##4}%
    {\normalfont\bfseries\,;\par}%
  }
  \newcommand{\res}{#1}
  \vspace{1ex}
  \noindent
  \encabezadoDeProc{#1}{#2}{#3}{#4}
  % Abrimos la llave
  \{\par%
  \tocarEspacios
}
% Ahora viene el cierre del ambiente problema
{
  % Cerramos la llave
  \noindent\}
  \vspace{1ex}
}


  \newcommand{\aux}[4]{%
    {\normalfont\bfseries\ttfamily aux\ }%
    {\normalfont\ttfamily #1}%
    \ifthenelse{\equal{#2}{}}{}{\ (#2)}\ : #3\, = \ensuremath{#4}%
    {\normalfont\bfseries\,;\par}%
  }


% \newcommand{\pre}[1]{\textsf{pre}\ensuremath{(#1)}}

\newcommand{\procnom}[1]{\textsf{#1}}
\newcommand{\procil}[3]{\textsf{proc #1}\ensuremath{(#2) = #3}}
\newcommand{\procilsinres}[2]{\textsf{proc #1}\ensuremath{(#2)}}
\newcommand{\preil}[2]{\textsf{Pre #1: }\ensuremath{#2}}
\newcommand{\postil}[2]{\textsf{Post #1: }\ensuremath{#2}}
\newcommand{\auxil}[2]{\textsf{fun }\ensuremath{#1 = #2}}
\newcommand{\auxilc}[4]{\textsf{fun }\ensuremath{#1( #2 ): #3 = #4}}
\newcommand{\auxnom}[1]{\textsf{fun }\ensuremath{#1}}
\newcommand{\auxpred}[3]{\textsf{pred }\ensuremath{#1( #2 ) \{ #3 \}}}

\newcommand{\comentario}[1]{{/*\ #1\ */}}

\newcommand{\nom}[1]{\ensuremath{\mathsf{#1}}}


% En las practicas/parciales usamos numeros arabigos para los ejercicios.
% Aca cambiamos los enumerate comunes para que usen letras y numeros
% romanos
\newcommand{\arreglarincisos}{%
  \renewcommand{\theenumi}{\alph{enumi}}
  \renewcommand{\theenumii}{\roman{enumii}}
  \renewcommand{\labelenumi}{\theenumi)}
  \renewcommand{\labelenumii}{\theenumii)}
}



%%%%%%%%%%%%%%%%%%%%%%%%%%%%%% PARCIAL %%%%%%%%%%%%%%%%%%%%%%%%
\let\@xa\expandafter
\newcommand{\tituloparcial}{\centerline{\depto -- \lamateria}
  \centerline{\elnombre -- \lafecha}%
  \setlength{\TPHorizModule}{10mm} % Fija las unidades de textpos
  \setlength{\TPVertModule}{\TPHorizModule} % Fija las unidades de
                                % textpos
  \arreglarincisos
  \newcounter{total}% Este contador va a guardar cuantos incisos hay
                    % en el parcial. Si un ejercicio no tiene incisos,
                    % cuenta como un inciso.
  \newcounter{contgrilla} % Para hacer ciclos
  \newcounter{columnainicial} % Se van a usar para los cline cuando un
  \newcounter{columnafinal}   % ejercicio tenga incisos.
  \newcommand{\primerafila}{}
  \newcommand{\segundafila}{}
  \newcommand{\rayitas}{} % Esto va a guardar los \cline de los
                          % ejercicios con incisos, asi queda mas bonito
  \newcommand{\anchode
  }{20} % Es para textpos
  \newcommand{\izquierda}{7} % Estos dos le dicen a textpos donde colocar
  \newcommand{\abajo}{2}     % la grilla
  \newcommand{\anchodecasilla}{0.4cm}
  \setcounter{columnainicial}{1}
  \setcounter{total}{0}
  \newcounter{ejercicio}
  \setcounter{ejercicio}{0}
  \renewenvironment{ejercicio}[1]
  {%
    \stepcounter{ejercicio}\textbf{\noindent Ejercicio \theejercicio. [##1
      puntos]}% Formato
    \renewcommand\@currentlabel{\theejercicio}% Esto es para las
                                % referencias
    \newcommand{\invariante}[2]{%
      {\normalfont\bfseries\ttfamily invariante}%
      \ ####1\hspace{1em}####2%
    }%
    \newcommand{\Proc}[5][result]{
      \encabezadoDeProc{####1}{####2}{####3}{####4}\hspace{1em}####5}%
  }% Aca se termina el principio del ejercicio
  {% Ahora viene el final
    % Esto suma la cantidad de incisos o 1 si no hubo ninguno
    \ifthenelse{\equal{\value{enumi}}{0}}
    {\addtocounter{total}{1}}
    {\addtocounter{total}{\value{enumi}}}
    \ifthenelse{\equal{\value{ejercicio}}{1}}{}
    {
      \g@addto@macro\primerafila{&} % Si no estoy en el primer ej.
      \g@addto@macro\segundafila{&}
    }
    \ifthenelse{\equal{\value{enumi}}{0}}
    {% No tiene incisos
      \g@addto@macro\primerafila{\multicolumn{1}{|c|}}
      \bgroup% avoid overwriting somebody else's value of \tmp@a
      \protected@edef\tmp@a{\theejercicio}% expand as far as we can
      \@xa\g@addto@macro\@xa\primerafila\@xa{\tmp@a}%
      \egroup% restore old value of \tmp@a, effect of \g@addto.. is

      \stepcounter{columnainicial}
    }
    {% Tiene incisos
      % Primero ponemos el encabezado
      \g@addto@macro\primerafila{\multicolumn}% Ahora el numero de items
      \bgroup% avoid overwriting somebody else's value of \tmp@a
      \protected@edef\tmp@a{\arabic{enumi}}% expand as far as we can
      \@xa\g@addto@macro\@xa\primerafila\@xa{\tmp@a}%
      \egroup% restore old value of \tmp@a, effect of \g@addto.. is
      % global
      % Ahora el formato
      \g@addto@macro\primerafila{{|c|}}%
      % Ahora el numero de ejercicio
      \bgroup% avoid overwriting somebody else's value of \tmp@a
      \protected@edef\tmp@a{\theejercicio}% expand as far as we can
      \@xa\g@addto@macro\@xa\primerafila\@xa{\tmp@a}%
      \egroup% restore old value of \tmp@a, effect of \g@addto.. is
      % global
      % Ahora armamos la segunda fila
      \g@addto@macro\segundafila{\multicolumn{1}{|c|}{a}}%
      \setcounter{contgrilla}{1}
      \whiledo{\value{contgrilla}<\value{enumi}}
      {%
        \stepcounter{contgrilla}
        \g@addto@macro\segundafila{&\multicolumn{1}{|c|}}
        \bgroup% avoid overwriting somebody else's value of \tmp@a
        \protected@edef\tmp@a{\alph{contgrilla}}% expand as far as we can
        \@xa\g@addto@macro\@xa\segundafila\@xa{\tmp@a}%
        \egroup% restore old value of \tmp@a, effect of \g@addto.. is
        % global
      }
      % Ahora armo las rayitas
      \setcounter{columnafinal}{\value{columnainicial}}
      \addtocounter{columnafinal}{-1}
      \addtocounter{columnafinal}{\value{enumi}}
      \bgroup% avoid overwriting somebody else's value of \tmp@a
      \protected@edef\tmp@a{\noexpand\cline{%
          \thecolumnainicial-\thecolumnafinal}}%
      \@xa\g@addto@macro\@xa\rayitas\@xa{\tmp@a}%
      \egroup% restore old value of \tmp@a, effect of \g@addto.. is
      \setcounter{columnainicial}{\value{columnafinal}}
      \stepcounter{columnainicial}
    }
    \setcounter{enumi}{0}%
    \vspace{0.2cm}%
  }%
  \newcommand{\tercerafila}{}
  \newcommand{\armartercerafila}{
    \setcounter{contgrilla}{1}
    \whiledo{\value{contgrilla}<\value{total}}
    {\stepcounter{contgrilla}\g@addto@macro\tercerafila{&}}
  }
  \newcommand{\grilla}{%
    \g@addto@macro\primerafila{&\textbf{TOTAL}}
    \g@addto@macro\segundafila{&}
    \g@addto@macro\tercerafila{&}
    \armartercerafila
    \ifthenelse{\equal{\value{total}}{\value{ejercicio}}}
    {% No hubo incisos
      \begin{textblock}{\anchodegrilla}(\izquierda,\abajo)
        \begin{tabular}{|*{\value{total}}{p{\anchodecasilla}|}c|}
          \hline
          \primerafila\\
          \hline
          \tercerafila\\
          \tercerafila\\
          \hline
        \end{tabular}
      \end{textblock}
    }
    {% Hubo incisos
      \begin{textblock}{\anchodegrilla}(\izquierda,\abajo)
        \begin{tabular}{|*{\value{total}}{p{\anchodecasilla}|}c|}
          \hline
          \primerafila\\
          \rayitas
          \segundafila\\
          \hline
          \tercerafila\\
          \tercerafila\\
          \hline
        \end{tabular}
      \end{textblock}
    }
  }%
  \vspace{0.4cm}
  \textbf{Nro. de orden:}

  \textbf{LU:}

  \textbf{Apellidos:}

  \textbf{Nombres:}
  
  \textbf{Nro. de hojas que adjunta:}
  \vspace{0.5cm}
}



% AMBIENTE CONSIGNAS
% Se usa en el TP para ir agregando las cosas que tienen que resolver
% los alumnos.
% Dentro del ambiente hay que usar \item para cada consigna

\newcounter{consigna}
\setcounter{consigna}{0}

\newenvironment{consignas}{%
  \newcommand{\consigna}{\stepcounter{consigna}\textbf{\theconsigna.}}%
  \renewcommand{\ejercicio}[1]{\item ##1 }
  \renewcommand{\proc}[5][result]{\item
    \encabezadoDeProc{##1}{##2}{##3}{##4}\hspace{1em}##5}%
  \newcommand{\invariante}[2]{\item%
    {\normalfont\bfseries\ttfamily invariante}%
    \ ##1\hspace{1em}##2%
  }
  \renewcommand{\aux}[4]{\item%
    {\normalfont\bfseries\ttfamily aux\ }%
    {\normalfont\ttfamily ##1}%
    \ifthenelse{\equal{##2}{}}{}{\ (##2)}\ : ##3 \hspace{1em}##4%
  }
  % Comienza la lista de consignas
  \begin{list}{\consigna}{%
      \setlength{\itemsep}{0.5em}%
      \setlength{\parsep}{0cm}%
    }
}%
{\end{list}}



% para decidir si usar && o ^
\newcommand{\y}[0]{\ensuremath{\land}}

% macros de correctitud
\newcommand{\semanticComment}[2]{#1 \ensuremath{#2};}
\newcommand{\namedSemanticComment}[3]{#1 #2: \ensuremath{#3};}


\newcommand{\local}[1]{\semanticComment{local}{#1}}

\newcommand{\vale}[1]{\semanticComment{vale}{#1}}
\newcommand{\valeN}[2]{\namedSemanticComment{vale}{#1}{#2}}
\newcommand{\impl}[1]{\semanticComment{implica}{#1}}
\newcommand{\implN}[2]{\namedSemanticComment{implica}{#1}{#2}}
\newcommand{\estado}[1]{\semanticComment{estado}{#1}}

\newcommand{\invarianteCN}[2]{\namedSemanticComment{invariante}{#1}{#2}}
\newcommand{\invarianteC}[1]{\semanticComment{invariante}{#1}}
\newcommand{\varianteCN}[2]{\namedSemanticComment{variante}{#1}{#2}}
\newcommand{\varianteC}[1]{\semanticComment{variante}{#1}}

\usepackage{caratula} % Version modificada para usar las macros de algo1 de ~> https://github.com/bcardiff/dc-tex
\usepackage{xcolor}

% === macros === %
\newcommand{\type}[2]{%
    {\normalfont\bfseries\ttfamily\noindent type\ }%
    {\normalfont\ttfamily #1}
    {= #2\\}%
}
\newcommand{\enum}[2]{%
    {\normalfont\bfseries\ttfamily\noindent enum\ }%
    {\normalfont\ttfamily #1}
    {\{#2\}}%
}
\newcommand{\comment}[2]{%
    \\[0.5ex]
    {#1\comentario{\text{#2}}}
    \\[0.5ex]
}


\begin{document}

% === caratula === %
\titulo{TP de Especificaci\'on}
\subtitulo{An\'alisis Habitacional Argentino}
\fecha{8 de Septiembre de 2021}
\materia{Algoritmos y Estructuras de Datos I}
\grupo{Grupo 02, comisi\'on 11}
\newcommand{\senial}{\textit{se\~nal}}

\integrante{Lakowsky, Manuel}{511/21}{mlakowsky@gmail.com}
\integrante{Lenardi, Juan Manuel}{56/14}{juanlenardi@gmail.com}
\integrante{Arienti, Federico}{316/21}{fa.arienti@gmail.com}

\maketitle

% === problemas === %
\section{Problemas}

% === esEncuestaValida === %
\subsection{proc. esEncuestaValida}

    \subsubsection{Especificaci\'on:}
        \begin{proc}{esEncuestaValida}{\In th: $eph_{h}$, \In ti : $eph_{i}$, \Out result: \bool}{}
            \pre{\True}
            \post{res = \True \Iff validarEncuesta(th,\ ti)}
        \end{proc}

    \subsubsection{Predicados y funciones auxiliares:}
        \noindent\pred{validarEncuesta}{th: $eph_{h}$, ti: $eph_{i}$}{
            \comment{\tab}{tabla hogares}
            \tab (esTabla(th,\ @largoItemHogar)\ \yLuego\\
            \tab (\forall h: hogar)(h \in th\ \implicaLuego\ (\\
            \tab\tab codigoValido_h(th,\ ti,\ h)\ \y\ a\tilde{n}oyTrimestreCongruente_h(th,\ h)\ \y\ attEnRango_h(h)\\
            \tab))) \y
            \comment{\tab}{tabla individuos}
            \tab (esTabla(ti,\ @largoItemIndividuo)\ \yLuego\\
            \tab(\forall i: individuo)(i \in ti\ \implicaLuego\ (\\
            \tab\tab codigoValido_i(th,\ ti,\ i)\ \y\ a\tilde{n}oyTrimestreCongruente_i(th,\ i)\ \y\ attEnRango_i(i)\ \y\\
            \tab\tab validarComponente_i(ti,\ i)\\
            \tab)))\\
        }
        $\newline$
        \pred{codigoValido$_{h}$}{th: $eph_{h}$, ti: $eph_{i}$, h: $hogar$}{\\
            \tab(\exists i: individuo)(i \in ti\ \yLuego\\
            \tab\tab h[@hogcodusu] = i[@indcodusu]\\
            \tab) \y\\
            \tab\neg(\exists h_2: hogar)((h_2 \in th\ \y\ h_2 \neq h)\ \yLuego\\ 
            \tab\tab h[@hogcodusu] = h_2[@hogcodusu]\\
            \tab)\\
        }
        $\newline$
        \pred{añoyTrimestreCongruente$_{h}$}{th: $eph_{h}$, h: $hogar$}{\\
            \tab h[@hoga\tilde{n}o] = th[0][@hoga\tilde{n}o]\ \y \ h[@hogtrimestre] = th[0][@hogtrimestre]\\
        }
        $\newline$
        \pred{attEnRango$_{h}$}{h: $hogar$}{\\
            \tab 0 \leq h[@hogcodusu]\ \y\\ 
            \tab 1810 \leq h[@hoga\tilde{n}o]\ \y\\
            \tab 1 \leq h[@hogtrimestre] \leq 4\ \y\\
            \tab -90 \leq h[@hoglatitud] \leq 90\ \y\\
            \tab -180 \leq h[@hoglongitud] \leq 180\ \y\\
            \tab 1 \leq h[@ii7] \leq 3\ \y\\ 
            \tab 1 \leq h[@region] \leq 6\ \y\\ 
            \tab 0 \leq h[@mas\_500] \leq 1\ \y\\
            \tab 1 \leq h[@iv1] \leq 5\ \y\\ 
            \tab 0 < h[@ii2] \leq h[@iv2]\ \y\\
            \tab 1 \leq h[@ii3] \leq 2\\
        }
        $\newline$          
        \noindent\pred{codigoValido$_{i}$}{th: $eph_{h}$, ti: $eph_{i}$, i: $individuo$}{\\
            \tab(\exists h: hogar)(h \in th\ \yLuego \\
            \tab\tab i[@indcodusu] = h[@hogcodusu]\\
            \tab) \y\\
            \tab\neg(\exists i_2: individuo)((i_2 \in ti\ \y\  i_2 \neq i)\ \yLuego\ (\\
            \tab\tab i[@indcodusu] = i_2[@indcodusu]\ \y\ i[@componente] = i_2[@componente]\\
            \tab))\\
        }
        $\newline$
        \pred{añoyTrimestreCongruente$_{i}$}{th: $eph_{h}$, i: $individuo$}{\\
            \tab i[@inda\tilde{n}o] = th[0][@hoga\tilde{n}o]\ \y\ i[@indtrimestre] = th[0][@hogtrimestre]\\
        }
        $\newline$
        \pred{attEnRango$_{i}$}{i: $individuo$}{\\
            \tab 0 \leq i[@indcodusu]\ \y\\ 
            \tab 1 \leq i[@componente] \leq 20\ \y\\
            \tab 1810 \leq i[@inda\tilde{n}o]\ \y\\
            \tab 1 \leq i[@indtrimestre] \leq 4\ \y\\
            \tab 1 \leq i[@ch4] \leq 2\ \y\\
            \tab 0 \leq i[@ch6]\ \y\\
            \tab 0 \leq i[@nivel\_ed] \leq 1\ \y\\ 
            \tab -1 \leq i[@estado] \leq 1\ \y\\ 
            \tab 0 \leq i[@cat\_ocup] \leq 4\ \y\\ 
            \tab -1 \leq i[@p47t]\ \y\\
            \tab 1 \leq i[@ppo4g] \leq 10\\    
        }
        $\newline$
        \pred{validarComponente$_{i}$}{ti: $eph_{i}$, i: $individuo$}{\\
            \tab i[@componente] = 1\ \vee (\exists i_2: individuo)(i_2 \in ti\ \yLuego\ i[@componente] - 1 = i_2[@componente])\\
        }

    \subsubsection{Observaciones:}
        \begin{itemize}
            \item se hace uso de diversos tipos y referencias definidos en 2.3 y 2.4.
            \item la funci\'on auxiliar $esTabla$, definida en 2.1., verifica que th y ti sean matrices del largo correcto y 
            con al menos una entrada.
            \item los predicados $codigoValido$ verifican, de forma cruzada, que los hogares tengan individuos asociados y viceversa, 
            y que no est\'en repetidos.    
            \item los predicados $a\tilde{n}oyTrimestreCongruente$ contrastan con la primer entrada de la tabla de hogares para asegurar
            la homogeneidad de los registros. 
            \item el predicado $validarComponente_{i}$ junto a $codigoValido_{i}$, y aplicado a todo individuo de la tabla, verifica que los 
            componentes ocurran de forma continua, es decir sin saltos mayores a 1, a partir del primero. 
            En consecuencia, basta con verificar \'estos predicados, y que los componentes est\'en en el rango correcto para asegurar 
            que no haya m\'as de 20 individuos por hogar. 
            \item consideramos que: 
                \begin{itemize}
                    \item $@hogcodusu$ y $@indcodusu$ son estrictamente positivos.
                    \item $@componente$ puede tomar valores entre 1 y 20 inclusive.
                    \item $@hoga\tilde{n}o$ y $@inda\tilde{n}o$ no pueden ser anteriores a la revoluci\'on de mayo.
                    \item $@hogtrimestre$ y $@indtrimestre$ toman valores entre 1 y 4 inclusive.
                    \item $@hoglatitud$ representa la direcci\'on $sur$ con n\'umeros negativos y $norte$ con positivos.
                    \item $@hoglongitud$ representa la direcci\'on $oeste$ con n\'umeros negativos y $este$ con positivos.
                    \item $@ch6$, al representar la edad, es mayor o igual a 0.
                    \item $@iv2$, la cantidad total de ambientes, es estrictamente mayor a 0.
                \end{itemize}
        \end{itemize}

% === histHabitacional === %   
\pagebreak
\subsection{proc. histHabitacional}

    \subsubsection{Especificaci\'on:}
        \begin{proc}{histHabitacional}{\In th: $eph_{h}$, \In ti: $eph_{i}$, \In region: $dato$, \Out res: \TLista{\ent}}{}
        \pre{validarEncuesta(th,\ ti)\ \y\ hayCasasEnLaRegion(th,\ region)}
        % CORREGIR
        \post{\\
            \textcolor{red}{HayMaximoDeHabitaciones(res)}\ \yLuego \\
            (\forall i:\ent)(0\leq i < |res|\ \implicaLuego\\ 
            \tab res[i] = \#casasPorNroDeHabitaciones(th,\ k,\ i + 1)\\
        )}
        \end{proc}

    \subsubsection{Predicados y funciones auxiliares:}
        \noindent\pred{hayCasasEnLaRegion}{th: $eph_{h}$, region: $dato$}{\\
            \tab(\exists h: hogar)(h \in th\ \yLuego\ esHogarValido_{1{.}2}(h,\ region))\\
        }
        $\newline$
        \noindent\pred{esHogarValido$_{1{.}2}$}{h: $hogar$, region: $dato$}{\\
            \tab h[@region] = region\ \y\ h[@iv1] = 1\\
        }
        $\newline$
    % CORREGIR
    \textcolor{red}{
        \noindent\pred{HayMaximoDeHabitaciones}{res: \TLista{\ent}}{\\
            \tab(\exists max: \ent)(max = |res|\ \y\ res[max-1] > 0)\\
        }
    }
        $\newline$
        \noindent\aux{$\#$casasPorNroDeHabitaciones}{th: $eph_{h}$, region: $dato$, habitaciones: \ent}{\ent}{\\[2ex]
            \tab\displaystyle\sum_{h \in th}
            {(\IfThenElse {esHogarValido_{1{.}2}(h,\ region)\ \y\ h[@iv2] = habitaciones}{1}{0})}
        }

    \subsubsection{Observaciones:}
        \begin{itemize}
            \item se hace uso del predicado $validarEncuesta$ definido en 1.1.2.
            \item consideramos, mediante el predicado $hayCasasEnLaRegion$, que no tiene sentido preguntarse sobre el 
            histograma habitacional de una regi\'on si \'esta no tiene hogares.
        \end{itemize}


% === laCasaEstaQuedandoChica === % 
\pagebreak
\subsection{proc. laCasaEstaQuedandoChica}

    \subsubsection{Especificaci\'on:}
        \begin{proc}{laCasaEstaQuedandoChica}{\In th: $eph_{h}$, \In ti: $eph_{i}$, \Out res: \TLista{\float}}{}
            \pre{validarEncuesta(th,\ ti)}
            \post{|res| = 6\ \yLuego\ (\forall region: dato)(1 \leq region \leq 6\ \implicaLuego\ res[region - 1] = \%hacinado(th,\ ti,\ region))}
        \end{proc}

    \subsubsection{Predicados y funciones auxiliares:}
        \noindent\pred{$\Omega$NoVacio$_{1{.}3}$}{th: $eph_{h}$, region: $dato$}{\\
            \tab(\exists h: $hogar$)(h \in th\ \yLuego\ esHogarValido_{1{.}3}(h,\ region))\\
        }
        $\newline$
        \noindent\pred{esHogarValido$_{1{.}3}$}{h: $hogar$, region: $dato$}{\\
            \tab h[@region] = region\ \y\ h[@mas\_500] = 0\ \y\ h[@iv1] = 1\\
        }
        $\newline$
        \noindent\pred{casaHacinada}{ti: $eph_{i}$, h: $hogar$, region: $dato$}{\\
            \tab esHogarValido_{1{.}3}(h, region)\ \y\ \#individuosEnHogar(ti,\ h[@hogcodusu]) > 3 * h[@iv2]\\
        }
        $\newline$
        \noindent\aux{$\%$hacinado}{th: $eph_{h}$, ti: $eph_{i}$, region: $dato$}{\float}{\\[2ex]
            \tab\IfThenElse{{\Omega}NoVacio_{1{.}3}(th,\ region)}{
                \frac{\displaystyle\sum_{h \in th}{(\IfThenElse{casaHacinada(ti,\ h,\ region)}{1}{0})}}
                    {\displaystyle\sum_{h \in th}{(\IfThenElse{esHogarValido_{1{.}3}(h,\ region)}{1}{0})}}
            }{0}
        }
    
    \subsubsection{Observaciones:}
        \begin{itemize}
            \item se hace uso de la funci\'on auxiliar ${\#}individuosEnHogar$ definida en 2.2.
            \item la funci\'on auxiliar ${\%}hacinado$ considera como espacio de probabilidad ($\Omega$) a todos los hogares que cumplan 
            con el predicado $esHogarValido_{1{.}3}$.
        \end{itemize}


% === creceElTeleworkingEnCiudadesGrandes === %
\pagebreak
\subsection{proc. creceElTeleworkingEnCiudadesGrandes}

    \subsubsection{Especificaci\'on:}
        \begin{proc}{creceElTeleworkingEnCiudadesGrandes}{\In t1h: $eph_{h}$, \In t1i: $eph_{i}$, \In t2h: $eph_{h}$, \In t2i: $eph_{i}$, \Out res: \bool}{}
            \pre{
                (validarEncuesta(t1h,\ t1i)\ \y\ validarEncuesta(t2h,\ t2i))\ \yLuego\ comparacionValida(t1h,\ t1i,\ t2h,\ t2i)
            }
            \post{res = \True \iff $\%$teleworking(t1h,\ t1i) < $\%$teleworking(t2h,\ t2i)}
        \end{proc}

    \subsubsection{Predicados y funciones auxiliares:}
        \noindent\pred{comparacionValida}{t1h: $eph_{h}$, t1i: $eph_{i}$, t2h: $eph_{h}$, t2i: $eph_{i}$}{\\
            \tab (t1h[0][@hoga\tilde{n}o] = t2h[0][@hoga\tilde{n}o] - 1\ \y\ t1h[0][@hogtrimestre] = t2h[0][@hogtrimestre])\\
            % \tab t1h[0][@hoga\tilde{n}o] < t2h[0][@hoga\tilde{n}o] \y \\    
        }
        $\newline$
        \pred{$\Omega$NoVacio$_{1{.}4}$}{th: $eph_{h}$}{\\
            \tab(\exists h:\ $hogar$)(h \in th\ \yLuego\ esHogarValido_{1{.}4}(h))\\
        }
        $\newline$
        \pred{esHogarValido$_{1{.}4}$}{h: $hogar$}{\\
            %\comment{\tab}{Hogar cumple con especificaciones}
            \tab h[@mas\_500] = 1\ \y\ (h[@iv1] = 1\ \vee\ h[@iv1] = 2)\\
        }
        $\newline$
        \pred{haceTeleworking}{th: $eph_{h}$, i: $individuo$}{\\
            %\comment{\tab}{Hogar e Individuo cumplen con especificaciones}
            \tab viveEnHogarValido(th, i)\ \y\ i[@ii3] = 1\ \y\ i[@ppo4g] = 6\\
        }
        $\newline$
        \pred{viveEnHogarValido}{th: $eph_{h}$, i: $individuo$}{\\
            %\comment{\tab}{Hogar del individuo cumple con especificaciones}
            \tab esHogarValido_{1{.}4}(th[indiceHogarPorCodusu(th,\ i[@indcodusu])])\\
        }
        $\newline$
        \aux{$\%$teleworking}{th: $eph_{h}$, ti: $eph_{i}$}{\float}{\\[2ex]
            \tab\IfThenElse{{\Omega}NoVacio_{1{.}4}(th)}{
                \frac{
                    \displaystyle\sum_{i \in ti}(\IfThenElse{haceTeleworking(th,\ i)}{1}{0})
                }{
                    \displaystyle\sum_{i \in ti}(\IfThenElse{viveEnHogarValido(th,\ i)}{1}{0})
                }
            }{0}
        }
            
    \subsubsection{Observaciones:}
        \begin{itemize}
            \item se hace uso del predicado $indiceHogarPorCodusu$ definido en 2.2. bajo la presunci\'on de una encuesta v\'alida.
            \item consideramos como comparaci\'on v\'alida a aquella realizada entre encuestas de años consecutivos.
            \item la funci\'on auxiliar ${\%}teleworking$ considera como espacio de probabilidad ($\Omega$) a todos los individuos que cumplan 
            con el predicado $viveEnHogarValido$.
        \end{itemize}


% === costoSubsidioMejora === % 
\pagebreak
\subsection{proc. costoSubsidioMejora}

    \subsubsection{Especificaci\'on:}    
        \begin{proc}{costoSubsidioMejora}{\In th: $eph_{i}$, \In ti: $eph_{i}$, \In monto: \ent, \Out res: \ent}{}
            \pre{validarEncuesta(th,\ ti) \y monto > 0} 
            \post{res = monto * \displaystyle\sum_{h \in th}(\IfThenElse{esHogarValido_{1{.}5}(ti,\ h)}{1}{0})}    
        \end{proc}

    \subsubsection{Predicados y funciones auxiliares:}
        \noindent\pred{esHogarValido$_{1{.}5}$}{ti: $eph_{i}$, h: $hogar$}{\\
            \tab h[@ii7] = 1\ \y\ h[@iv1] = 1\ \y\ \#individuosEnHogar(ti,\ h[@hogcodusu]) - 2 > h[@ii2]\\
        }

    \subsubsection{observaciones:}
        \begin{itemize}
            \item consideramos que un subsidio es necesariamente un monto positivo y que, dado el objetivo final de la especificaci\'on
            debe ser mayor a 0.
        \end{itemize}


% === generales === %
\pagebreak
\section{Predicados y Auxiliares generales}
                    
\subsection{Predicados Generales}
    \noindent\pred{esMatriz}{s: \TLista{\TLista{T}}}{\\
        \tab(\forall fila: \TLista{T})(fila \in s\ \implicaLuego\ |fila| = |s[0]|)\\
    }
    $\newline$
    \noindent\pred{esTabla}{m: \TLista{\TLista{T}}, columnas: \ent}{\\
        \tab |m| > 0 \yLuego (|m[0]| = columnas \y esMatriz(m))\\
    }

\subsection{Auxiliares Generales}     
    \noindent\aux{$\#$individuosEnHogar}{ti: $eph_{i}$, codusu$_{h}$: $dato$}{\ent}{
        \displaystyle\sum_{i \in ti}(\IfThenElse {i[@indcodusu] = codusu_h}{1}{0})
    }
    $\comment{}{indiceHogarPorCodusu asume codusu$_{h}$ existe en la tabla y es único}$
    \noindent\aux{indiceHogarPorCodusu}{th: $eph_{h}$, codusu$_{h}$: $dato$}{\ent}{
        \displaystyle\sum_{h \in th}\IfThenElse{h[@hogcodusu] = codusu_h}{i}{0}
    }

\subsection{Tipos y Enumerados}
    \type {dato}{\ent}
    \type {individuo}{\TLista{dato}} 
    \type {hogar}{\TLista{dato}}
    \type {eph$_i$}{\TLista{individuo}}
    \type {eph$_h$}{\TLista{hogar}}
    \type {joinHI}{\TLista{hogar \times individuo}}

    \enum {ItemHogar}{hogcodusu, hogaño, hogtrimestre, hoglatitud, hoglongitud, ii7, region, mas\_500, iv1, iv2, ii2, ii3} 
    \\
    \enum {ItemIndividuo}{indcodusu, componente, indaño, indtrimestre, ch4, ch6, nivel\_ed, cat\_ocup, p47t, ppo4g}

\subsection{Referencias}
    \noindent\aux{@hogcodusu}{}{\ent}{itemHogar.ord(hogcodusu)}
    \noindent\aux{@hogaño}{}{\ent}{itemHogar.ord(hoga\tilde{n}o)}
    \noindent\aux{@hogtrimestre}{}{\ent}{itemHogar.ord(hogtrimestre)}
    \noindent\aux{@hoglatitud}{}{\ent}{itemHogar.ord(hoglatitud)}
    \noindent\aux{@hoglongitud}{}{\ent}{itemHogar.ord(hoglongitud)}
    \noindent\aux{@ii7}{}{\ent}{itemHogar.ord(ii7)}
    \noindent\aux{@region}{}{\ent}{itemHogar.ord(region)}
    \noindent\aux{@mas\_500}{}{\ent}{itemHogar.ord(mas\_500)}
    \noindent\aux{@iv1}{}{\ent}{itemHogar.ord(iv1)}
    \noindent\aux{@iv2}{}{\ent}{itemHogar.ord(iv2)}
    \noindent\aux{@ii2}{}{\ent}{itemHogar.ord(ii2)}
    \noindent\aux{@ii3}{}{\ent}{itemHogar.ord(ii3)}
    $\newline$
    \noindent\aux{@indcodusu}{}{\ent}{itemIndividuo.ord(indcodusu)}
    \noindent\aux{@componente}{}{\ent}{itemIndividuo.ord(componente)}
    \noindent\aux{@indaño}{}{\ent}{itemIndividuo.ord(inda\tilde{n}o)}
    \noindent\aux{@indtrimestre}{}{\ent}{itemIndividuo.ord(indtrimestre)}
    \noindent\aux{@ch4}{}{\ent}{itemIndividuo.ord(ch4)}
    \noindent\aux{@ch6}{}{\ent}{itemIndividuo.ord(ch6)}
    \noindent\aux{@nivel\_ed}{}{\ent}{itemIndividuo.ord(nivel\_ed)}
    \noindent\aux{@cat\_ocup}{}{\ent}{itemIndividuo.ord(cat\_ocup)}
    \noindent\aux{@p47t}{}{\ent}{itemIndividuo.ord(p47t)}
    \noindent\aux{@ppo4g}{}{\ent}{itemIndividuo.ord(ppo4g)}
    $\newline$
    \noindent\aux{@largoItemHogar}{}{\ent}{12}
    \noindent\aux{@largoitemIndividuo}{}{\ent}{10}

\end{document}

