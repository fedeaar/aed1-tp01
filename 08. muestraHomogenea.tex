% === muestraHomogenea === %
\subsection{proc. muestraHomogenea}
\subsubsection{Especificaci\'on:}
        \begin{proc}{muestraHomogenea}{\In th : $eph_{h}$, \In ti : $eph_{i}$, \Out res : \TLista{hogar}}{}
            \pre{sonTablasValidas(th,\ ti)}
            \post{\\
                ((\exists s:\TLista{hogar})(esLaSecuenciaMasLarga(th,\ ti,\ s)) \y res = s) \vee\\
                (\neg(\exists s:\TLista{hogar})(esLaSecuenciaMasLarga(th,\ ti,\ s)) \y |res| = 0)\\ 
            }
            %\post{\IfThenElse{(\exists s:\TLista{hogar})(esLaSecuenciaMasLarga(th,\ ti,\ s))}{res = s}{|res| = 0}}
            %\post{esLaSecuenciaMasLarga(th,\ ti,\ res) \vee |res| = 0 }
            
            % obs: no se si vale usar res = seq[], creo que más correcto sería decir |res| = 0
           
            % CORREGIDO: En el post, eliminación de un predicado y replanteo del otro disjunto. 
            
        \end{proc}
 
    \subsubsection{Predicados y funciones auxiliares:}
    % cosas de formato:
    % aca propongo ordenar por orden de llamado: ej. empezar con esLaSecuenciaMasLarga, le sigue lo que se llame en ese pred, 
    % y despues seguir con diferenciaConstanteDeIngresos.
    % detalle de formato, cambio las mayusculas iniciales por minusculas, siguiendo el resto del tp. Es por una convención, se 
    % suelen poner solo a las clases (un tipo de objeto que no vimos aun) con mayuscula inicial, para distinguirlas. 
    % arreglo tambien, algunas cositas de espaciado.
     
    \noindent\pred{esLaSecuenciaMasLarga}{th : $eph_{h}$,\ ti :  $eph_{i}$,\ res : \TLista{hogar}}{\\
      \tab esSecuenciaHomogenea(th,\ ti,\ res)\ \y\\ 
      \tab\neg(\exists s : \TLista{hogar})(esSecuenciaHomogenea(th,\ ti,\ s)\ \y\ |s| > |res|)\\
    }
    $\newline$
    \noindent\pred{esSecuenciaHomogenea}{th : $eph_{h}$,\ ti :  $eph_{i}$,\ res : \TLista{hogar}}{\\
        \tab |res| \geq 3\ \y\\ 
        \tab contieneHogaresValidos(th,\ res)\ \y\\ 
        \tab ordenCrecienteEntreIngresos(ti,\ res)\ \y\\ 
        \tab laDiferenciaEsConstante(ti,\ res)\\
    }
     $\newline$
    % obs: faltan los tipos en los parametros de los pred
    \noindent\pred{contieneHogaresValidos}{th : $eph_{h}$,\ res : \TLista{hogar}}{\\
        % obs: creo que habria que nombrarla distinto, ya que chequea que todos los hogares de res esten en th, no solo uno
        % algo tal vez como: contieneHogaresValidos (?)
        % RESPUESTA: sugerencia tomada
        \tab (\forall i:\ent)(0 \leq i < |res|\ \implicaLuego\ res[i] \in th)\\
    }
    $\newline$
    \noindent\pred{ordenCrecienteEntreIngresos}{ti : $eph_{i}$,\ res : \TLista{hogar}}{\\
        \tab (\forall i:\ent)(0 \leq i < |res| - 1\ \implicaLuego\ diferenciaEntreIngresosConsecutivos(ti,\ res,\ i) \geq 0)\\
        % obs: al decir que es >= 0, como sabes que entre pares consecutivos sigue habiendo orden creciente?
        % por otro lado, el orden creciente creo que tiene que ser respecto al ingreso particular de cada hogar, no a la diferencia
        % RESPUESTA: No se esta ordenando restos, la resta ordena los restando. Se cambio el nombre anterior, "ordenCreciente".
    }
    $\newline$
    \noindent\aux{diferenciaEntreIngresosConsecutivos}{ti : $eph_{i}$,\ res : \TLista{hogar},\ i : \ent}{\ent}{\\[1ex]
        % obs: aca tal vez un nombre mas declarativo sea diferenciaDeIngresos(EntreHogaresConsecutivos?) o ingresosConsecutivos
        % o algo por el estilo
        % RESPUESTA: Se cambio "consecutivos" por "diferenciaEntreIngresosConsecutivos"
        \tab ingresoPorHogar(ti,\ res[i + 1]) - ingresoPorHogar(ti,\ res[i])
    }
    $\newline$
    \noindent\aux{ingresoPorHogar}{ti : $eph_{i}$,\ h : $hogar$}{\ent}{\\[2ex]
        % obs: aca tal vez le quedaria mejor un nombre como : sumaDeIngresosPorHogar 
        % RESPUESTA: El ingreso por hogar es la suma de sus ingresos. No se desea hacer referencia a como se obtiene ese total.   
        % Sugiero que conserve el nombre
        \tab\displaystyle\sum_{i \in ti}{\IfThenElse{i[@indcodusu] = h[@hogcodusu]}{i[@p47T]}{0}}
        % cambio ind por i para homogeneizar 
    }
    \vspace*{2ex}
    \noindent\pred{laDiferenciaEsConstante}{ti : $eph_{i}$,\ res : \TLista{hogar}}{\\
        % obs: otro nombre podria ser ingresosConsecutivosIguales (?)
        % RESPUESTA: se cambio nombre original, "consecutivosIguales"
        \tab (\forall i:\ent)(0 \leq i < |res| - 2\ \implicaLuego\\ 
        \tab\tab diferenciaEntreIngresosConsecutivos(ti,\ res,\ i) =\\
        \tab\tab\tab diferenciaEntreIngresosConsecutivos(ti,\ res,\ i + 1)\\
        \tab)\\
    }
   
    \subsubsection{Observaciones:}
        \begin{itemize}
            \item 
            % posibles comentarios:
            
            \item decidimos denotar el resultado de la secuencia vacía, en la postcondición, por medio de una de sus propiedades: 
            que su largo sea igual a cero.
            \item en el predicado $esLaSecuenciaMasLarga$ consideramos que, si fueran a haber dos, o más, secuencias del mismo largo 
            que cumplan con la especificación, esto no debería invalidar a ninguna de ellas de ser una respuesta válida.
            \item el predicado $ordenCrecienteEntreIngresos$ utiliza el hecho de que la $diferenciaEntreIngresosConsecutivos$ es 
            positiva solo si los ingresos del hogar $i + 1$ son mayores o iguales a los del hogar $i$.
            \item el predicado $laDiferenciaEsConstante$ evalúa hasta $|res| - 2$, ya que $diferenciaEntreIngresosConsecutivos$ 
            evalúa hasta un indice más de aquel con el que es llamado.
        \end{itemize}
