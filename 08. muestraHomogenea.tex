% === muestraHomogenea === %
\subsection{proc. muestraHomogenea}
\subsubsection{Especificaci\'on:}
        \begin{proc}{muestraHomogenea}{\In th : $eph_{h}$, \In ti : $eph_{i}$, \Out res : \TLista{hogar}}{}
            \pre{sonTablasValidas(th,\ ti)}
            \post{(esLaSecuenciaMasLarga\ \y\ diferenciaConstanteDeIngresos(th,\ ti,\ res)) \vee  res=\TLista{}}
            % obs: no se si vale usar res = seq[], creo que más correcto sería decir |res| = 0
        \end{proc}
 
    \subsubsection{Predicados y funciones auxiliares:}
    % cosas de formato:
    % aca propongo ordenar por orden de llamado: ej. empezar con esLaSecuenciaMasLarga, le sigue lo que se llame en ese pred, 
    % y despues seguir con diferenciaConstanteDeIngresos.
    % detalle de formato, cambio las mayusculas iniciales por minusculas, siguiendo el resto del tp. Es por una convención, se 
    % suelen poner solo a las clases (un tipo de objeto que no vimos aun) con mayuscula inicial, para distinguirlas. 
    % arreglo tambien, algunas cositas de espaciado.

    \noindent\aux{ingresoPorHogar}{ti,\ h : $hogar$}{\ent}{\\[2ex]
        % obs: aca tal vez le quedaria mejor un nombre como : sumaDeIngresosPorHogar 
        \tab\displaystyle\sum_{i \in ti}{(\IfThenElse{i[@indcodusu] = h[@hogcodusu]}{i[@p47T]}{0})}
        % cambio ind por i para homogeneizar 
    }
    \vspace*{2ex}
    \noindent\aux{consecutivos}{ti,\ res,\ i : \ent}{\ent}{\\[2ex]
        % obs: aca tal vez un nombre mas declarativo sea diferenciaDeIngresos(EntreHogaresConsecutivos?) o ingresosConsecutivos
        % o algo por el estilo
        \tab ingresoPorHogar(ti,\ res[i + 1]) - ingresoPorHogar(ti,\ res[i])
    }
    $\newline$
    % obs: faltan los tipos en los parametros de los pred
    \noindent\pred{esHogar}{th,\ res}{\\
        % obs: creo que habria que nombrarla distinto, ya que chequea que todos los hogares de res esten en th, no solo uno
        % algo tal vez como: contieneHogaresValidos (?)
        \tab (\forall i:\ent)(0 \leq i < |res|\ \y\ res[i] \in th)\\
    }
    $\newline$
    \noindent\pred{ordenCreciente}{ti,\ res}{\\
        \tab (\forall i:\ent)(0 \leq i < |res| - 1\ \implica\ consecutivos(ti,\ res,\ i) \geq 0)\\
        % obs: al decir que es >= 0, como sabes que entre pares consecutivos sigue habiendo orden creciente?
        % por otro lado, el orden creciente creo que tiene que ser respecto al ingreso particular de cada hogar, no a la diferencia 
    }
    $\newline$
    \noindent\pred{consecutivosIguales}{ti,\ res}{\\
        % obs: otro nombre podria ser ingresosConsecutivosIguales (?)
        \tab (\forall i:\ent)(0 \leq i < |res| - 2\ \implica\ consecutivos(ti,\ res,\ i + 1) = consecutivos(ti,\ res,\ i))\\
    }
    $\newline$
    \noindent\pred{diferenciaConstanteDeIngresos}{ti,\ th,\ res}{\\
        \tab (esHogar(th,\ res)\ \y\ ordenCreciente(ti,\ res)\ \y\ consecutivosIguales(ti,\ res))\\
    }
    $\newline$
    \noindent\pred{esLaSecuenciaMasLarga}{}{\\
    }
    
    \subsubsection{Observaciones:}
        \begin{itemize}
            % aca comentaria sobre la logica de los indices en ordenCreciente y consecutivosIguales
            \item
        \end{itemize}
