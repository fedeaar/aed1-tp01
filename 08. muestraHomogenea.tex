% === muestraHomogenea === %
\subsection{proc. muestraHomogenea}
\subsubsection{Especificaci\'on:}
        \begin{proc}{muestraHomogenea}{\In th : $eph_{h}$, \In ti : $eph_{i}$, \Out res : \TLista{hogar}}{}
            \pre{sonTablasValidas(th,\ ti)}
            \post{esLaSecuenciaMasLarga(th,\ ti,\ res) \vee |res| = 0 }
            
            % obs: no se si vale usar res = seq[], creo que más correcto sería decir |res| = 0
           
            % CORREGIDO: En el post, eliminación de un predicado y replanteo del otro disjunto. 
            
        \end{proc}
 
    \subsubsection{Predicados y funciones auxiliares:}
    % cosas de formato:
    % aca propongo ordenar por orden de llamado: ej. empezar con esLaSecuenciaMasLarga, le sigue lo que se llame en ese pred, 
    % y despues seguir con diferenciaConstanteDeIngresos.
    % detalle de formato, cambio las mayusculas iniciales por minusculas, siguiendo el resto del tp. Es por una convención, se 
    % suelen poner solo a las clases (un tipo de objeto que no vimos aun) con mayuscula inicial, para distinguirlas. 
    % arreglo tambien, algunas cositas de espaciado.
     
    \noindent\pred{esLaSecuenciaMasLarga}{th,ti,res}{\\
      \tab esSecuenciaHomogenia(ti,th,res) \y\ \neg(\exists s \in \Tlista{hogar})(esSecuenciaHomogenia(ti,th,s) \y\ |s|\geq|res|)
    }
    $\newline$
    \noindent\pred{esSecuenciaHomogenia}{ti,\ th,\ res}{\\
        \tab (|res|\geq 3\ \y\ contieneHogaresValidos(th,\ res)\ \y\ ordenCrecienteEntreIngresos(ti,\ res)\ \y\ 
        diferenciaConstanteEntreIngresosConsecutivos(ti,\ res))\\
    }
     $\newline$
    % obs: faltan los tipos en los parametros de los pred
    \noindent\pred{contieneHogaresValidos}{th,\ res}{\\
        % obs: creo que habria que nombrarla distinto, ya que chequea que todos los hogares de res esten en th, no solo uno
        % algo tal vez como: contieneHogaresValidos (?)
        % RESPUESTA: sugerencia tomada
        \tab (\forall i:\ent)(0 \leq i < |res|\ \y\ res[i] \in th)\\
    }
    $\newline$
    \noindent\pred{ordenCrecienteEntreIngresos}{ti,\ res}{\\
        \tab (\forall i:\ent)(0 \leq i < |res| - 1\ \implica\ diferenciaEntreIngresosConsecutivos(ti,\ res,\ i) \geq 0)\\
        % obs: al decir que es >= 0, como sabes que entre pares consecutivos sigue habiendo orden creciente?
        % por otro lado, el orden creciente creo que tiene que ser respecto al ingreso particular de cada hogar, no a la diferencia
        % RESPUESTA: No se esta ordenando restos, la resta ordena los restando. Se cambio el nombre anterior, "ordenCreciente".
    }
    $newline$
     \vspace*{2ex}
    \noindent\aux{diferenciaEntreIngresosConsecutivos}{ti,\ res,\ i : \ent}{\ent}{\\[2ex]
        % obs: aca tal vez un nombre mas declarativo sea diferenciaDeIngresos(EntreHogaresConsecutivos?) o ingresosConsecutivos
        % o algo por el estilo
        % RESPUESTA: Se cambio "consecutivos" por "diferenciaEntreIngresosConsecutivos"
        \tab ingresoPorHogar(ti,\ res[i + 1]) - ingresoPorHogar(ti,\ res[i])
    }
    $\newline$
    \noindent\aux{ingresoPorHogar}{ti,\ h : $hogar$}{\ent}{\\[2ex]
        % obs: aca tal vez le quedaria mejor un nombre como : sumaDeIngresosPorHogar 
        % RESPUESTA: El ingreso por hogar es la suma de sus ingresos. No se desea hacer referencia a como se obtiene ese total.   
        % Sugiero que conserve el nombre
        \tab\displaystyle\sum_{i \in ti}{(\IfThenElse{i[@indcodusu] = h[@hogcodusu]}{i[@p47T]}{0})}
        % cambio ind por i para homogeneizar 
    }
    $\newline$
    \noindent\pred{diferenciaConstanteEntreIngresosConsecutivos}{ti,\ res}{\\
        % obs: otro nombre podria ser ingresosConsecutivosIguales (?)
        % RESPUESTA: se cambio nombre original, "consecutivosIguales"
        \tab (\forall i:\ent)(0 \leq i < |res| - 2\ \implica\ diferenciaEntreIngresosConsecutivos(ti,\ res,\ i + 1) = diferenciaEntreIngresosConsecutivos(ti,\ res,\ i))\\
    }
   
   
   
    
    
   
    \subsubsection{Observaciones:}
        \begin{itemize}
            % aca comentaria sobre la logica de los indices en ordenCreciente y consecutivosIguales
            \item
        \end{itemize}
