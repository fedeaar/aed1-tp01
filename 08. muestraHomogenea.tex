% === muestraHomogenea === %
\subsection{proc. muestraHomogenea}
\subsubsection{Especificaci\'on:}
        \begin{proc}{muestraHomogenea}{\In th: $eph_{h}$, \In ti: $eph_{i}$, \Out res: \TLista{hogar}}{}
            \pre{sonTablasValidas(th, ti)}
            \post{(EsLaSecuenciaMasLarga \y DiferenciaConstanteDeIngresos(th, ti, res)) \vee  res=\Tlista{}}
        \end{proc}
 
    \subsubsection{Predicados y funciones auxiliares:}
    \aux{IngresoPorHogar}{ti,h : hogar}{\ent}{\\
        \tab\displaystyle\sum_{ind \in ti}{(\IfThenElse{ind[@indcodusu]=h[@hogcodusu]}{ind[@p47T]}{0})}
    }
    $newline$
    \aux{Consecutivos}{ti,res,i : \ent}{\ent}{\\
        \tab IngresoPorHogar(ti,res[i+1])-IngresoPorHogar(ti,res[i])
    }
    $newline$
    \noindent\pred{EsHogar}{th, res}{\\
        \tab (\forall i:\ent)(0 \leq i < |res| \y res[i] \in th)
    }
    $newline$
    \noindent\pred{OrdenCreciente}{ti, res}{\\
        \tab (\forall i:\ent)(0 \leq i < |res|-1 \luego Consecutivos(ti,res,i) \geq 0)
    }
    $newline$
    \noindent\pred{ConsecutivosIguales}{ti, res}{\\
        \tab (\forall i:\ent)(0 \leq i < |res|-2 \luego Consecutivos(ti,res,i+1) = Consecutivos(ti,res,i) )
    }
    $newline$
    \noindent\pred{DiferenciaConstanteDeIngresos}{ti,th,res}{\\
        \tab (EsHogar(th,res) \y OrdenCreciente(ti,res) \y ConsecutivosIguales(ti,res))
    }
    $newline$
    \noindent\pred{EsLaSecuenciaMasLarga}{}{\\
    }
    
    \subsubsection{Observaciones:}
        \begin{itemize}
            \item
        \end{itemize}
