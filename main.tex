\documentclass[a4paper]{article} 

\setlength{\parskip}{0.1em}
\newcommand{\tab}[1][1.2cm]{\hspace*{#1}}
\input{Algo1Macros}
\usepackage{caratula} % Version modificada para usar las macros de algo1 de ~> https://github.com/bcardiff/dc-tex
\usepackage{xcolor}

% === macros === %
\newcommand{\type}[2]{%
    {\normalfont\bfseries\ttfamily\noindent type\ }%
    {\normalfont\ttfamily #1}
    {= #2\\}%
}
\newcommand{\enum}[2]{%
    {\normalfont\bfseries\ttfamily\noindent enum\ }%
    {\normalfont\ttfamily #1}
    {\{#2\}}%
}
\newcommand{\comment}[2]{%
    \\[0.5ex]
    {#1\comentario{\text{#2}}}
    \\[0.5ex]
}

% === documento === %
\begin{document}


% === caratula === %
\titulo{TP de Especificaci\'on}
\subtitulo{An\'alisis Habitacional Argentino}
\fecha{8 de Septiembre de 2021}
\materia{Algoritmos y Estructuras de Datos I}
\grupo{Grupo 02, comisi\'on 11}
\newcommand{\senial}{\textit{se\~nal}}

\integrante{Lakowsky, Manuel}{511/21}{mlakowsky@gmail.com}
\integrante{Lenardi, Juan Manuel}{56/14}{juanlenardi@gmail.com}
\integrante{Arienti, Federico}{316/21}{fa.arienti@gmail.com}

\maketitle


% === problemas === %
\section{Problemas}

\input{1. esEncuestaValida.tex}  
\pagebreak

\input{2. histHabitacional.tex}  
\pagebreak

% === laCasaEstaQuedandoChica === % 
\subsection{proc. laCasaEstaQuedandoChica}

    \subsubsection{Especificaci\'on:}
        \begin{proc}{laCasaEstaQuedandoChica}{\In th: $eph_{h}$, \In ti: $eph_{i}$, \Out res: \TLista{\float}}{}
            \pre{sonEncuestasValidas(th,\ ti)\ \y\ 1 \leq region \leq 6}
            \post{|res| = 6\ \yLuego\ (\forall region: dato)(1 \leq region \leq 6\ \implicaLuego\ res[region - 1] = \%hacinado(th,\ ti,\ region))}
        \end{proc}

    \subsubsection{Predicados y funciones auxiliares:}
        \noindent\pred{$\Omega$NoVacio$_{1{.}3}$}{th: $eph_{h}$, region: $dato$}{\\
            \tab(\exists h: $hogar$)(h \in th\ \yLuego\ esHogarValido_{1{.}3}(h,\ region))\\
        }
        $\newline$
        \noindent\pred{esHogarValido$_{1{.}3}$}{h: $hogar$, region: $dato$}{\\
            \tab h[@region] = region\ \y\ h[@mas\_500] = 0\ \y\ h[@iv1] = 1\\
        }
        $\newline$
        \noindent\pred{casaHacinada}{ti: $eph_{i}$, h: $hogar$, region: $dato$}{\\
            \tab esHogarValido_{1{.}3}(h, region)\ \y\ \#individuosEnHogar(ti,\ h[@hogcodusu]) > 3 * h[@iv2]\\
        }
        $\newline$
        \noindent\aux{$\%$hacinado}{th: $eph_{h}$, ti: $eph_{i}$, region: $dato$}{\float}{\\[2ex]
            \tab\IfThenElse{{\Omega}NoVacio_{1{.}3}(th,\ region)}{
                \frac{\displaystyle\sum_{h \in th}{(\IfThenElse{casaHacinada(ti,\ h,\ region)}{1}{0})}}
                    {\displaystyle\sum_{h \in th}{(\IfThenElse{esHogarValido_{1{.}3}(h,\ region)}{1}{0})}}
            }{0}
        }
    
    \subsubsection{Observaciones:}
        \begin{itemize}
            \item se hace uso de la funci\'on auxiliar ${\#}individuosEnHogar$ definida en 2.2.
            \item la funci\'on auxiliar ${\%}hacinado$ considera como espacio de probabilidad ($\Omega$) a todos los hogares que cumplan 
            con el predicado $esHogarValido_{1{.}3}$.
            \item en el predicado $\%hacinado$ consideramos que si no hay hogares v\'alidos en una región, entonces la proporción de hogares hacinados respecto a esa región es 0.
        \end{itemize}
\pagebreak

\input{4. creceElTeleworkingEnCiudadesGrandes.tex}
\pagebreak

\input{5. costoSubsidioMejora.tex}
\pagebreak

\input{6. generarJoin.tex}
\pagebreak

\input{7. ordenarRegionYTipo.tex}
\pagebreak

\input{8. muestraHomogenea.tex}
\pagebreak

% === corregirRegion === %
\subsection{proc. corregirRegion}
    \subsubsection{Especificaci\'on:}    
    \subsubsection{Predicados y funciones auxiliares:}
    \subsubsection{Observaciones:}
        \begin{itemize}
            \item
        \end{itemize}
\pagebreak

% === histogramaDeAnillosConcentricos === %
\subsection{proc. histogramaDeAnillosConcentricos}
    \subsubsection{Especificaci\'on:}    
        \begin{proc}{histogramaDeAnillosConcentricos}{\In th: $eph_{h}$, \In centro: \ent$\times$\ent, \In distancias: \TLista{\ent}, \Out res: \TLista{\ent}}{}
            \pre{esTablaDeHogaresValida(th)\ \y\ esCentroValido(centro)\ \y\ sonDistanciasValidas(distancias)} 
            \post{\\
            |result| = |distancias|\ \yLuego (\\
            result[0] = \#HogaresEnAnillo(th,\ centro,\ 0,\ distancias[0])\ \y\\
            (\forall i:\ent)(0 < i < |result|\ \implicaLuego\\ 
                \tab result[i] = \#HogaresEnAnillo(th,\ centro,\ distancias[i - 1],\ distancias[i])\\
            ))} 
        \end{proc}
    
    \subsubsection{Predicados y funciones auxiliares:}
        \noindent\pred{esCentroValido}{centro \ent$\times$\ent}{\\
        \tab -90 \leq centro_{0} \leq 90\ \y\ -180 \leq centro_{1} \leq 180\\
        }
        $\newline$
        \noindent\pred{sonDistanciasValidas}{distancias: \TLista{\ent}}{\\
            \tab|distancias| > 0\ \yLuego\ (distancias[0] > 0\ \y\ (\forall i:\ent)(0 \leq i < |distancias| - 1\ \implicaLuego\ distancias[i] < distancias[i + 1]))\\
        }
        $\newline$
        \aux{$\#$HogaresEnAnillo}{th :  $eph_{h}$, centro: \ent$\times$\ent, desde: \ent, hasta: \ent}{\ent}{\\[2ex]
            \tab\displaystyle\sum_{h \in th}\IfThenElse{cuadrado(desde) \leq distancia(h,\ centro) < cuadrado(hasta)}{1}{0}
        }
        $\newline$
        \aux{distancia}{h: $hogar$, centro: \ent$\times$\ent}{\float}{
            cuadrado(h[@hoglatitud] - centro_{0}) + cuadrado(h[@hoglongitud] - centro_{1})
        }
        $\newline$
        \aux{cuadrado}{n : \ent}{\ent}{n * n}

    \subsubsection{Observaciones:}
        \begin{itemize}
            \item se hace uso del predicado $esTablaDeHogaresValida$ definido en 1.1.2.
            \item Dado que la pertenencia de una distancia P = (x, y) a un anillo concéntrico definido en el intervalo (positivo) de radios [A, B) respecto al centro C = (x$_{0}$, y$_{0}$) se define como: 
                \begin{equation}
                    A\ \leq\ \sqrt{(x - x_{0})^{2} + (y - y_{0})^{2}}\ <\ B
                \end{equation}
                Por simple manipulación algebráica (elevando al cuadrado), la misma relación se mantiene para:
                \begin{equation}
                    A^{2}\ \leq\ (x - x_{0})^{2} + (y - y_{0})^{2}\ <\ B^{2}
                \end{equation}
                el predicado $\#HogaresEnAnillo$ hace uso de esta observaci\'on.
                
                % agregar que todo es nec positivo
                % y otras obs
        \end{itemize}
\pagebreak


% === quitarIndividuos === %
\subsection{proc. quitarIndividuos}
    
    \subsubsection{Especificaci\'on:}    
        
        \begin{proc}{quitarIndividuos}{\Inout th : $eph_{h}$, \Inout ti : $eph_{i}$, \In busqueda : \TLista{(ItemIndividuo,\ dato)}, \Out result : ($eph_{h}$,\ $eph_{i}$)}{}
            \pre{
                sonEncuestasValidas(th,\ ti)\ \y\ esBusquedaValida(busqueda)\ \y\ th = th_{0}\ \y\ ti = ti_{0}
            }
            \post{\\
                (esParticion(th_{0},\ th,\ result_{0})\ \y\ esParticion(ti_{0},\ ti,\ result_{1}))\ \yLuego\\
                (losIndividuosEstanFiltrados(ti_{0},\ ti,\ result_{1},\ busqueda)\ \y\\ 
                losHogaresEstanFiltrados(th_{0},\ th,\ result_{0},\ ti_{0},\ busqueda))\\
            }
        \end{proc}

    \subsubsection{Predicados y funciones auxiliares:}
        
        \noindent\pred{esBusquedaValida}{busqueda : \TLista{(ItemIndividuo,\ dato)}}{\\
            % del mismo modo que no hace falta validar que un argumento se un entero, no hace falta validar que es un itemIndividuo
            \tab(\forall i : \ent)(0 \leq i < |busqueda| \implicaLuego (\\
            \tab\tab pideUnDatoValido(busqueda[i])\ \y\\
            \tab\tab\neg(\exists j : \ent)((0 \leq i < |busqueda|\ \y\ i \neq j)\ \yLuego\ (busqueda[i])_{0} = (busqueda[j])_{0})\\
            \tab))\\
        }
        $\newline$
        \pred{pideUnDatoValido}{condicion : (ItemIndividuo,\ dato)}{\\
            \tab (condicion_{0} = indcodusu\ \y\ 0 \leq condicion_{1})\ \vee\\
            \tab (condicion_{0} = componente\ \y\ 1 \leq condicion_{1} \leq 20)\ \vee\\
            \tab (condicion_{0} = inda\tilde{n}o\ \y\ 1810 \leq condicion_{1})\ \vee\\
            \tab (condicion_{0} = indtrimestre\ \y\ 1 \leq condicion_{1} \leq 4)\ \vee\\
            \tab (condicion_{0} = ch4\ \y\ 1 \leq condicion_{1} \leq 2)\ \vee\\
            \tab (condicion_{0} = ch6\ \y\ 0 \leq condicion_{1})\ \vee\\
            \tab (condicion_{0} = nive\_ed\ \y\ 0 \leq condicion_{1} \leq 1)\ \vee\\
            \tab (condicion_{0} = estado\ \y\ -1 \leq condicion_{1} \leq 1)\ \vee\\
            \tab (condicion_{0} = cat\_ocup\ \y\ 0 \leq condicion_{1} \leq 4)\ \vee\\
            \tab (condicion_{0} = p47t\ \y\ -1 \leq condicion_{1})\ \vee\\
            \tab (condicion_{0} = ppo4g\ \y\ 1 \leq condicion_{1} \leq 10)\\
        }
        $\newline$
        \pred{esParticion}{original, sub$_{1}$, sub$_{2}$ : \TLista{T}}{\\
            %\tab |original| \geq |sub_{1}|\ \y\ |sub_{2}| = |original| - |sub_{1}|\ \y\\
            \tab(\forall i:T)(i \in sub_{1} ++\ sub_{2}\ \iff i \in original)\\ 
            %\y\ \neg(\exists i : T)(i \in sub_{1}\ \y\ i \in sub_{2})\\ 
        }
        $\newline$
        \pred{losIndividuosEstanFiltrados}{ti$_{0}$, ti, result$_{1}$, : $eph_{i}$, busqueda : \TLista{(ItemIndividuo,\ dato)}}{\\
        \tab(\forall i : $individuo$)(i \in ti_{0}\ \implicaLuego\ (\\
        \tab\tab (i \in result_{1}\ \y\ i \notin ti)\ \iff esBusquedaExitosa(i,\ busqueda)\\
        \tab))\\
        }
        $\newline$
        \pred{losHogaresEstanFiltrados}{th$_{0}$, th, result$_{0}$, : $eph_{h}$, ti$_{0}$ : $eph_{i}$,  busqueda : \TLista{(ItemIndividuo,\ dato)}}{\\
        \tab(\forall h : $hogar$)(h \in th_0\ \implicaLuego\ (\\
        \tab\tab (h \in result_{0}\ \y\ h \notin th)\ \iff\\
        \tab\tab (\forall i : $individuo$)((i \in ti_0\ \y\ i[@indcodusu] = h[@hogcodusu])\ \implicaLuego\ (\\
        \tab\tab\tab  esBusquedaExitosa(i,\ busqueda)\\
        \tab\tab))\\
        \tab))\\
        }
        $\newline$
        \pred{esBusquedaExitosa}{i : $individuo$, busqueda : \TLista{(ItemIndividuo,\ dato)}}{\\
            \tab (\forall\ condicion : (ItemIndividuo,\ dato))(condicion \in busqueda\ \implicaLuego\ (\\
            \tab\tab i[itemIndividuo.ord(condicion_{0})] = condicion_{1}\\ 
            \tab))\\
        }    
    
    
    \subsubsection{Observaciones:}
        \begin{itemize}
            \item
        \end{itemize}
\pagebreak

\input{0. generales.tex}

\end{document}

