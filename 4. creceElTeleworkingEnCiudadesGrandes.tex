% === creceElTeleworkingEnCiudadesGrandes === %
\subsection{proc. creceElTeleworkingEnCiudadesGrandes}

    \subsubsection{Especificaci\'on:}
        \begin{proc}{creceElTeleworkingEnCiudadesGrandes}{\In t1h: $eph_{h}$, \In t1i: $eph_{i}$, \In t2h: $eph_{h}$, \In t2i: $eph_{i}$, \Out res: \bool}{}
            \pre{
                (sonEncuestasValidas(t1h,\ t1i)\ \y\ sonEncuestasValidas(t2h,\ t2i))\ \yLuego\ comparacionValida(t1h,\ t1i,\ t2h,\ t2i)
            }
            \post{res = \True \iff $\%$teleworking(t1h,\ t1i) < $\%$teleworking(t2h,\ t2i)}
        \end{proc}

    \subsubsection{Predicados y funciones auxiliares:}
        \noindent\pred{comparacionValida}{t1h: $eph_{h}$, t1i: $eph_{i}$, t2h: $eph_{h}$, t2i: $eph_{i}$}{\\
            %\tab (t1h[0][@hoga\tilde{n}o] = t2h[0][@hoga\tilde{n}o] - 1\ \y\ 
            \tab t1h[0][@hogtrimestre] = t2h[0][@hogtrimestre])\ \y\ t1h[0][@hoga\tilde{n}o] < t2h[0][@hoga\tilde{n}o]\\
        }
        $\newline$
        \pred{$\Omega$NoVacio$_{1{.}4}$}{th: $eph_{h}$}{\\
            \tab(\exists h:\ $hogar$)(h \in th\ \yLuego\ esHogarValido_{1{.}4}(h))\\
        }
        $\newline$
        \pred{esHogarValido$_{1{.}4}$}{h: $hogar$}{\\
            %\comment{\tab}{Hogar cumple con especificaciones}
            \tab h[@mas\_500] = 1\ \y\ (h[@iv1] = 1\ \vee\ h[@iv1] = 2)\\
        }
        $\newline$
        \pred{haceTeleworking}{th: $eph_{h}$, i: $individuo$}{\\
            %\comment{\tab}{Hogar e Individuo cumplen con especificaciones}
            \tab viveEnHogarValido(th, i)\ \y\ i[@ii3] = 1\ \y\ i[@ppo4g] = 6\\
        }
        $\newline$
        \pred{viveEnHogarValido}{th: $eph_{h}$, i: $individuo$}{\\
            %\comment{\tab}{Hogar del individuo cumple con especificaciones}
            \tab esHogarValido_{1{.}4}(th[indiceHogarPorCodusu(th,\ i[@indcodusu])])\\
        }
        $\newline$
        \aux{$\%$teleworking}{th: $eph_{h}$, ti: $eph_{i}$}{\float}{\\[2ex]
            \tab\IfThenElse{{\Omega}NoVacio_{1{.}4}(th)}{
                \frac{
                    \displaystyle\sum_{i \in ti}(\IfThenElse{haceTeleworking(th,\ i)}{1}{0})
                }{
                    \displaystyle\sum_{i \in ti}(\IfThenElse{viveEnHogarValido(th,\ i)}{1}{0})
                }
            }{0}
        }
            
    \subsubsection{Observaciones:}
        \begin{itemize}
            \item se hace uso del predicado $indiceHogarPorCodusu$ definido en 2.2. bajo la presunci\'on de una encuesta v\'alida.
            %\item consideramos como comparaci\'on v\'alida a aquella realizada entre encuestas de años consecutivos.
            \item la funci\'on auxiliar ${\%}teleworking$ considera como espacio de probabilidad ($\Omega$) a todos los individuos que cumplan 
            con el predicado $viveEnHogarValido$.
            \item en el predicado $\%teleworking$ consideramos que si no hay hogares v\'alidos para considerar, entonces la proporción de hogares  respecto al total es 0.
        \end{itemize}

